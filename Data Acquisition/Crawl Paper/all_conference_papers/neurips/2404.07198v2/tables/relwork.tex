%\begin{table}[!htp]
%\centering
\vskip -0.25in
\caption{Comparison with existing \clqa approaches. \emph{Ind.} denotes inductive generalization to new entities \emph{(e)} and relations \emph{(r)}. \method is the first inductive method the generalizes to queries over new entities and relations at inference time.} \label{tab:rel_work}
%\scriptsize
\begin{adjustbox}{width=\textwidth}
\begin{tabular}{lcccc}\toprule
\textbf{Method} &\textbf{Ind. $e$} &\textbf{Ind. $r$} &\textbf{Ind. Logical Ops} \\\midrule
Query2Box~\cite{q2b}, BetaE~\cite{betae} & \nope & \nope & Parametric, \nope \\
CQD~\cite{cqd}, FuzzQE~\cite{fuzz_qe}, QTO~\cite{qto} & \nope & \nope & Fuzzy, \yes \\
GNN-QE~\cite{gnn_qe}, NodePiece-QE~\cite{galkin2022} & \yes & \nope &Fuzzy, \yes \\
\method \textbf{(this work)} & \yes & \yes &Fuzzy, \yes \\
\bottomrule
\end{tabular}
\end{adjustbox}
%\end{table}