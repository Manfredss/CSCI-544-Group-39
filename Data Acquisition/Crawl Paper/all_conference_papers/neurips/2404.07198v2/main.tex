\documentclass{article}

% if you need to pass options to natbib, use, e.g.:
\PassOptionsToPackage{numbers, compress}{natbib}
% before loading neurips_2024


% ready for submission
\usepackage[final]{neurips_2024}
% to compile a preprint version, e.g., for submission to arXiv, add add the
% [preprint] option:
%     \usepackage[preprint]{neurips_2024}

% to compile a camera-ready version, add the [final] option, e.g.:
%     \usepackage[final]{neurips_2024}

% to avoid loading the natbib package, add option nonatbib:
%    \usepackage[nonatbib]{neurips_2024}

% Recommended, but optional, packages for figures and better typesetting:
\usepackage{xspace}
\usepackage{xcolor}
\usepackage{microtype}
\usepackage{wrapfig}
\usepackage{graphicx}
\usepackage{subfigure}
\usepackage{enumitem}
\usepackage{multirow}
\usepackage{booktabs} % for professional tables
\usepackage{adjustbox}

\usepackage{hyperref}
\hypersetup{
    colorlinks,
    linkcolor={red!50!black},
    citecolor={blue!50!black},
    urlcolor={blue!80!black}
}


% Attempt to make hyperref and algorithmic work together better:
\newcommand{\theHalgorithm}{\arabic{algorithm}}

% For theorems and such
\usepackage{amsmath}
\usepackage{amssymb}
\usepackage{mathtools}
\usepackage{amsthm}

% if you use cleveref..
\usepackage[capitalize,noabbrev]{cleveref}

%%%%%%%%%%%%%%%%%%%%%%%%%%%%%%%%
% THEOREMS
%%%%%%%%%%%%%%%%%%%%%%%%%%%%%%%%
\theoremstyle{plain}
\newtheorem{theorem}{Theorem}[section]
\newtheorem{proposition}[theorem]{Proposition}
\newtheorem{lemma}[theorem]{Lemma}
\newtheorem{corollary}[theorem]{Corollary}
\theoremstyle{definition}
\newtheorem{definition}[theorem]{Definition}
\newtheorem{assumption}[theorem]{Assumption}
\theoremstyle{remark}
\newtheorem{remark}[theorem]{Remark}

% Todonotes is useful during development; simply uncomment the next line
%    and comment out the line below the next line to turn off comments
%\usepackage[disable,textsize=tiny]{todonotes}
\usepackage[textsize=tiny]{todonotes}

% ULTRA - Unified learnable and transferable KG representations
\newcommand{\method}{\textsc{UltraQuery}\xspace}
\newcommand{\methodlp}{\textsc{UltraQuery LP}\xspace}
\newcommand{\clqa}{CLQA\xspace}
\newcommand{\ultra}{\textsc{Ultra}\xspace}
\newcommand{\cqda}{$\text{CQD}^{\mathcal{A}}$\xspace}
\newcommand{\mathbbm}[1]{\text{\usefont{U}{bbm}{m}{n}#1}}
\newcommand{\ozero}{\textcircled{\scriptsize{0}}}
\newcommand{\oone}{\textcircled{\scriptsize{1}}}
\newcommand{\restatableeq}[3]{\label{#3}#2\gdef#1{#2\tag{\ref{#3}}}}
\newcommand{\ie}{\emph{i.e.}}
\newcommand{\eg}{\emph{e.g.}}

\newcommand{\graph}{\mathcal{G}}
\newcommand{\grel}{\mathcal{G}_r}
\newcommand{\gtrain}{\mathcal{G}_{\textit{train}}}
\newcommand{\ginf}{\mathcal{G}_{\textit{inf}}}
\newcommand{\ghat}{\hat{\mathcal{G}}_{\textit{inf}}}
\newcommand{\etrain}{\mathcal{V}_{\textit{train}}}
\newcommand{\einf}{\mathcal{V}_{\textit{inf}}}
\newcommand{\rtrain}{\mathcal{R}_{\textit{train}}}
\newcommand{\rinf}{\mathcal{R}_{\textit{inf}}}
\newcommand{\ttrain}{\mathcal{E}_{\textit{train}}}
\newcommand{\tinf}{\mathcal{E}_{\textit{inf}}}
\newcommand{\tpred}{\mathcal{E}_{\textit{pred}}}
\newcommand{\that}{\hat{\mathcal{E}}_{\textit{inf}}}

\newcommand{\ents}{\mathcal{V}}
\newcommand{\rels}{\mathcal{R}}
\newcommand{\edges}{\mathcal{E}}
\newcommand{\rfund}{\gR_{\textit{fund}}}
\newcommand{\vrfund}{\mR_{\textit{fund}}}

\newcommand{\zz}[1]{\textcolor{blue}{Zhaocheng: #1}}
\newcommand{\mg}[1]{\textcolor{purple}{MG: #1}}

% smaller citation font to be used in tables
% \makeatletter
% \DeclareRobustCommand\citeps
% {\begingroup\footnotesize\NAT@swatrue\let\NAT@ctype\z@\NAT@partrue
%     \@ifstar{\NAT@fulltrue\NAT@citetp}{\NAT@fullfalse\NAT@citetp}}
% \makeatother

\let\realcitet\citet
\newcommand*{\citets}[1]{{\footnotesize\realcitet{#1}}}

%\let\realcitet\citet
\newcommand*{\citeps}[1]{{\footnotesize\citep{#1}}}

\newcommand{\yes}{$\Large\color{teal}{\checkmark}$}
\newcommand{\nope}{$\Large\color{red}{\pmb\times}$}
\input{math_commands}

\title{A Foundation Model \\ for Zero-shot Logical Query Reasoning}
%\title{A Foundation Model for Zero-shot Logical Query Reasoning on any Knowledge Graph}


% The \author macro works with any number of authors. There are two commands
% used to separate the names and addresses of multiple authors: \And and \AND.
%
% Using \And between authors leaves it to LaTeX to determine where to break the
% lines. Using \AND forces a line break at that point. So, if LaTeX puts 3 of 4
% authors names on the first line, and the last on the second line, try using
% \AND instead of \And before the third author name.


\author{%
  Mikhail Galkin\textsuperscript{1}, Jincheng Zhou\textsuperscript{2,}\thanks{Work done during the internship at Intel. Code: \url{https://github.com/DeepGraphLearning/ULTRA}}, Bruno Ribeiro\textsuperscript{2}, Jian Tang\textsuperscript{3,4,5}, Zhaocheng Zhu\textsuperscript{3,6} \\
  \textsuperscript{1}Intel AI Lab, 
  \textsuperscript{2}Purdue University, 
  \textsuperscript{3}Mila - Qu\'ebec AI Institute, \\
  \textsuperscript{4}HEC Montr\'eal, \textsuperscript{5}CIFAR AI Chair
  \textsuperscript{6}Universit\'e de Montr\'eal 
  % examples of more authors
  % \And
  % Coauthor \\
  % Affiliation \\
  % Address \\
  % \texttt{email} \\
  % \AND
  % Coauthor \\
  % Affiliation \\
  % Address \\
  % \texttt{email} \\
  % \And
  % Coauthor \\
  % Affiliation \\
  % Address \\
  % \texttt{email} \\
  % \And
  % Coauthor \\
  % Affiliation \\
  % Address \\
  % \texttt{email} \\
}

\begin{document}

\maketitle

\vspace{-2em}
\begin{abstract}
Complex logical query answering (CLQA) in knowledge graphs (KGs) goes beyond simple KG completion and aims at answering compositional queries comprised of multiple projections and logical operations.
Existing CLQA methods that learn parameters bound to certain entity or relation vocabularies can only be applied to the  graph they are trained on which requires substantial training time before being deployed on a new graph.
%Here we present \method, an inductive reasoning model that can zero-shot answer logical queries on \emph{any} KG.
Here we present \method, the first foundation model for inductive reasoning that can zero-shot answer logical queries on \emph{any} KG.
The core idea of \method is to derive both projections and logical operations as vocabulary-independent functions which generalize to new entities and relations in any KG.
With the projection operation initialized from a pre-trained inductive KG 
%reasoning 
completion
model, \method can solve CLQA on any KG 
%even if it is only finetuned 
after finetuning
on a single dataset.
Experimenting on 23 datasets, \method in the zero-shot inference mode shows competitive or better query answering performance than best available baselines and sets a new state of the art on 15 of them.
\end{abstract}

\vspace{-1em}
\begin{figure}[!htbp]
    \centering
    \includegraphics[width=\linewidth]{figs/mainfig1_MRR_pyg.pdf}
    \vskip -0.12 in
    \caption{Zero-shot query answering performance (MRR, higher is better) of a single \method model trained on one FB15k237 queries dataset compared to the best available baselines and ablated \methodlp on 23 datasets. \emph{EPFO} is the average of 9 query types with $(\wedge, \lor)$ operators, \emph{Negation} is the average of 5 query types with the negation operator $(\neg)$. %-- trainable for each transductive and inductive $(e)$ dataset, and the heuristic baseline for newly introduced inductive $(e,r)$ datasets. 
    %\methodlp is an ablated version trained only on \emph{1p} link prediction. 
    On average, a single \method model outperforms the best baselines trained specifically on each dataset. More results are presented in Table~\ref{tab:maintab1} and Appendix~\ref{app:more_results}.
    }
    \label{fig:main_fig1}
\end{figure}

\section{Introduction}
\label{sec:intro}

\begin{wrapfigure}{R}{0.5\textwidth}
\begin{minipage}{0.5\textwidth}
\vspace{-1em}
    %\begin{figure}
        \centering
        \includegraphics[width=\textwidth]{figs/UltraQuery_Fig1.pdf}
        %\vspace{-2em}
         \vskip -0.1in
        \caption{The inductive logical query answering setup where training and inference graphs (and queries) have different entity and relation vocabularies. We propose a single model (\method) that zero-shot generalizes to query answering % against any unseen
    on any graph with new entity or relation vocabulary at inference time.}
        \label{fig:intro}
    %\end{figure}
    %\vskip -0.2in
    \vspace{-1em}
\end{minipage}
\end{wrapfigure}

Complex logical query answering (\clqa) generalizes simple knowledge graph (KG) completion to more complex, compositional queries with logical operators such as intersection $(\wedge)$, union $(\lor)$, and negation $(\lnot)$.
Such queries are expressed in a subset of first-order logic (FOL) where existentially quantified $(\exists)$ \emph{variables} and given \emph{constants} comprise \emph{relation projections} (or \emph{atoms}), and logical operators combine projections into a logical query (graph pattern). 
A typical example of a logical query~\citep{ren2023ngdb} is presented in \autoref{fig:intro}: $?U.\exists V : \texttt{Win}(\texttt{NobelPrize}, V) \land \texttt{Citizen}(\texttt{USA}, V) \land \texttt{Graduate}(V, U)$ where $\texttt{Win}()$ is a relation projection, $\texttt{NobelPrize}$ is a constant, and $V$ is an existentially quantified variable.




% \begin{figure}[t]
%     \centering
%     \includegraphics[width=0.5\columnwidth]{figs/UltraQuery_Fig1.pdf}
%     \vskip -0.1in
%     \caption{The inductive logical query answering setup where training and inference graphs (and queries) have different entity and relation vocabularies. We propose a single model (\method) that zero-shot generalizes to query answering % against any unseen
%     on any graph with new entity or relation vocabulary at inference time. \mg{wrapfig it into the first paragraph}}
%     \label{fig:intro}
%     \vskip -0.2in
% \end{figure}

% \clqa assumes that underlying KGs are incomplete such that the projection operator has to predict missing edges and reach the answers unobtainable by simple graph traversal.
Due to the incompleteness of most KGs, these logical queries cannot be directly solved by graph traversal algorithms. Consequently, \clqa methods have to deal with missing edges when modeling the projection operators.
The vast majority of existing \clqa methods~\citep{ren2023ngdb,q2b,betae,cqd,qto} % can only
predict missing edges by learning graph-specific entity and relation embeddings making such approaches transductive and not generalizable to other KGs. 
A few approaches~\citep{gnn_qe,galkin2022,sheaves} are able to generalize query answering to new nodes at inference time but still need a fixed relation vocabulary.

In this work, we focus on the hardest inductive generalization setup where queries and underlying graphs at inference time are completely different from the training graph, \ie, both entities and relations are new.   
Furthermore, we aim at performing \clqa in the \emph{zero-shot} setting with one single model. That is, instead of %training every model on each target dataset,
%adaptively 
finetuning a model on each target dataset,
we seek to design a unified approach that generalizes to any KG and %any complex 
query at inference time.
For example, in \autoref{fig:intro}, the training graph describes academic entities with relations \texttt{Win}, \texttt{Citizen}, \texttt{Graduate}\footnote{We assume the presence of respective inverse relations $\texttt{r}^{-1}$.} whereas the inference graph describes music entities with relations \texttt{Band Member} and \texttt{Plays}. 
The query against the inference graph $?U:\texttt{BandMember}(\texttt{Dire Straits}, U) \wedge \texttt{Plays}^{-1}(\texttt{Guitar}, U)$ involves both new entities 
%(\texttt{Dire Straits}, \texttt{Guitar}) 
and relations 
%(\texttt{Band Member}, \texttt{Plays}) 
and, to the best of our knowledge, cannot be tackled by any existing \clqa method that learns a fixed set of entities or relation embeddings from the training graph.

\textbf{Contributions.} 
Our contributions are two-fold. First, none of the existing \clqa methods can generalize to query answering over new arbitrary KGs with new entities and relations at inference time. 
We bridge this gap by leveraging the recent progress in inductive KG reasoning~\citep{ultra,isdea} and devise \method, the first %\clqa approach 
foundation model for \clqa that generalizes to logical queries on any arbitrary KG with any entity and relation vocabulary in the zero-shot fashion without relying on any external node or edge features. 
%\method follows the blueprint of GNN-QE~\citep{gnn_qe} with the  projection operator parameterized by a graph neural network (GNN) and non-parametric logical operators implemented with fuzzy logics~\citep{vankrieken_fuzzy}.
\method parameterizes the projection operator by an inductive graph neural network (GNN) and implements non-parametric logical operators with fuzzy logics~\citep{vankrieken_fuzzy}.
The pre-trained projection operator~\citep{ultra} does not learn any graph-specific entity nor relation embeddings thanks to the % transferable
generalizable meta-graph representation of relation interactions, and %is therefore zero-shot transferable to any KG.
therefore enables zero-shot generalization to any KG.

% Elaborate more on UltraQuery?

% Experimental evidence
% Furthermore,
Second, in the absence of existing datasets for our inductive generalization setup,
we curate a novel suite of 11 inductive query answering datasets where graphs and queries at inference time  have new entity and relation vocabularies.
Experimentally, we train a single \method model on one dataset and probe on other 22 transductive and inductive datasets.
Averaged across the datasets, a single \method model outperforms by 50\% (relative MRR) the best reported baselines in the literature (often tailored to specific graphs) on both EPFO queries and queries with negation.

\section{Related Work}
\label{sec:related_work}

\textbf{Complex Logical Query Answering.}
To the best of our knowledge, there is no existing approach for generalizable and inductive query answering where the method is required to deal with new entitis and relations at inference time.
% at inference time \emph{both} entities and relations are unseen.

Due to the necessity of learning entity and relation embeddings, the vast majority of existing methods like GQE~\citep{gqe}, BetaE~\citep{betae}, ConE~\citep{cone}, MPQE~\citep{mpqe} (and many more from the survey by \citet{ren2023ngdb}) are transductive-only and tailored for a specific set of entities and relations.
Among them, CQD~\citep{cqd} and QTO~\citep{qto} are inference-only query answering engines that execute logical operators with non-parametric fuzzy logic operators (\eg, product logic) but still require pre-trained entity and relation embedding matrices to execute relation projections (link prediction steps). 
We refer the interested reader to the comprehensive survey by \citet{ren2023ngdb} that covers query answering theory, a taxonomy of approaches, datasets, and open challenges.

\begin{wraptable}{R}{0.5\textwidth}
\begin{minipage}{0.5\textwidth}
    %\begin{table}[!htp]
%\centering
\vskip -0.25in
\caption{Comparison with existing \clqa approaches. \emph{Ind.} denotes inductive generalization to new entities \emph{(e)} and relations \emph{(r)}. \method is the first inductive method the generalizes to queries over new entities and relations at inference time.} \label{tab:rel_work}
%\scriptsize
\begin{adjustbox}{width=\textwidth}
\begin{tabular}{lcccc}\toprule
\textbf{Method} &\textbf{Ind. $e$} &\textbf{Ind. $r$} &\textbf{Ind. Logical Ops} \\\midrule
Query2Box~\cite{q2b}, BetaE~\cite{betae} & \nope & \nope & Parametric, \nope \\
CQD~\cite{cqd}, FuzzQE~\cite{fuzz_qe}, QTO~\cite{qto} & \nope & \nope & Fuzzy, \yes \\
GNN-QE~\cite{gnn_qe}, NodePiece-QE~\cite{galkin2022} & \yes & \nope &Fuzzy, \yes \\
\method \textbf{(this work)} & \yes & \yes &Fuzzy, \yes \\
\bottomrule
\end{tabular}
\end{adjustbox}
%\end{table}
\end{minipage}
\end{wraptable}


A few models~\citep{gnn_qe,galkin2022, sheaves} generalize only to % unseen entities at the cost of learning relation embedding matrices. 
new entities by modeling entities as a function of relation embeddings.
\citet{sheaves} apply the idea of cellular sheaves and harmonic extension to translation-based embedding models to answer conjunctive queries (without unions and negations).
NodePiece-QE~\citep{galkin2022} trains an inductive entity encoder (based on the fixed vocabulary of relations) that is able to reconstruct entity embeddings of the % unseen
new graph and then apply non-parametric engines like CQD to answer queries against % unseen
new entities.
The most effective inductive (entity) approach is GNN-QE~\cite{gnn_qe, galkin2022} that parameterizes each entity as a function of the relational structure between the query constants and the entity itself.
However, all these works rely on a fixed relation vocabulary and cannot generalize to KGs with new relations at test time.
In contrast, our model uses inductive relation projection and inductive logical operations that enable zero-shot generalization to any new KG with any entity and relation vocabulary without any specific training.
% The most effective inductive (entity) approach is GNN-QE~\citep{gnn_qe} that does not need entity embedding thanks to the inductive nature of the underlying NBFNet~\citep{nbfnet} encoder for relation projections and non-parametric fuzzy logic operators. 
% In this work, we leverage the \emph{learnable projection -- non-parametric logical operators} blueprint of GNN-QE and imbue the framework with a relation projection operator inductive to \emph{both} entities and relations that enables zero-shot generalization to any unseen KG with any entity and relation vocabulary without any specific training.

\textbf{Inductive Knowledge Graph Completion.}
In \clqa, KG completion methods execute the projection operator and are mainly responsible for predicting missing links in incomplete graphs during query execution. 
Inductive KG completion is usually categorized~\citep{chen2023generalizing} into two branches: (i) inductive entity (inductive $(e)$) approaches have a fixed set of relations and only generalize to % unseen
new entities, for example, to different subgraphs of one larger KG with one set of relations; and (ii) inductive entity and relation (inductive $(e,r)$) approaches that do not rely on any fixed set of entities and relations and generalize to any new KG with arbitrary new sets of entities and relations.

Up until recently, the majority of existing approaches belonged to the inductive $(e)$ family (\eg, GraIL~\citep{grail}, NBFNet~\citep{nbfnet}, RED-GNN~\citep{redgnn}, NodePiece~\citep{nodepiece}, A*Net~\citep{astarnet}, AdaProp~\citep{adaprop}) %, \emph{inter alia}) 
that generalizes only to % unseen
new entities as their featurization strategies are based on learnable relation embeddings. 

Recently, the more generalizable inductive $(e,r)$ family started getting more attention, \eg, with RMPI~\citep{rmpi}, InGram~\citep{ingram}, \ultra~\citep{ultra}, and the theory of \emph{double equivariance} introduced by \citet{isdea} followed by ISDEA and MTDEA~\citep{mtdea} models. 
% Consider moving the following to the main Method section.
In this work, we employ \ultra to obtain transferable graph representations and execute the projection operator with link prediction over any arbitrary KG without input features.
%Notably, \ultra does not require pre-computed input entity or edge features and relies solely on the graph structure.
%As different KGs might often have heterogeneous feature spaces, \eg, textual descriptions or numerical features of different dimensions, deriving a single fixed-width model for any arbitrary feature space is highly non-trivial and we deem this direction orthogonal to the present work. 
%As different KGs might have heterogeneous feature spaces, \eg, textual or numerical, 
Extending our model with additional input features is possible (although deriving a single fixed-width model for graphs with arbitrary input space % dimensions
is highly non-trivial) and we leave it for future work.
%as a promising avenue for exploration in future work.


\section{Preliminaries and Problem Definition}
\label{sec:prelim}

We introduce the basic concepts pertaining to logical query answering and KGs largely following the 
existing literature~\citep{galkin2022,ren2023ngdb,ultra}.%formalizations introduced in \citet{galkin2022}, \citet{ren2023ngdb}, and \citet{ultra}.

\textbf{Knowledge Graphs and Inductive Setup.}
Given a finite set of entities $\mathcal{V}$ (nodes), a finite set of relations $\mathcal{R}$ (edge types), and a set of triples (edges) $\mathcal{E} = (\mathcal{V} \times \mathcal{R} \times \mathcal{V})$, a knowledge graph $\mathcal{G}$ is a tuple $\mathcal{G} = (\mathcal{V}, \mathcal{R}, \mathcal{E})$.
The simplest \emph{transductive} setup dictates that the graph at training time $\gtrain = (\etrain, \rtrain, \ttrain)$ and the graph at inference (validation or test) time $\ginf = (\einf, \rinf, \tinf)$ are the same, \ie, $\gtrain = \ginf$. 
By default, we assume that the inference graph $\ginf$ is an incomplete part of a larger, non observable graph $\hat{\ginf}$ with missing triples to be predicted at inference time.
In the \emph{inductive} setup, in the general case, the training and inference graphs are different, $\gtrain \neq \ginf$.
In the easier inductive entity (\emph{inductive $(e)$}) setup tackled by most of the KG completion literature, the relation set $\mathcal{R}$ is fixed and shared between training and inference graphs, \ie, $\gtrain = (\etrain, \mathcal{R}, \ttrain)$ and $\ginf = (\einf, \mathcal{R}, \tinf)$. The inference graph can be an extension of the training graph if $\etrain \subseteq \einf$ or be a separate disjoint graph (with the same set of relations) if $\etrain \cap \einf = \varnothing$. 
In \clqa, the former setup with the extended training graph at inference is tackled by InductiveQE approaches~\citep{galkin2022}. 

In the hardest inductive entity and relation (inductive $(e,r)$) case, both entities and relations sets are different, \ie, $\etrain \cap \einf = \varnothing$ and $\rtrain \cap \rinf = \varnothing$. 
In \clqa, there is no existing approach tackling this case and our proposed \method is the first one to do so.

% In this work, we tackle this harder inductive (also known as \emph{fully-inductive}) case with both new, unseen entities and relation types at inference time. 
% Since the harder inductive case (with new relations at inference) is strictly a superset of the easier inductive scenario (with the fixed relation set), any model capable of fully-inductive inference is by design applicable in easier inductive scenarios as well.

\textbf{First-Order Logic Queries.}
Applied to KGs, a first-order logic (FOL) query $q$ is a formula that consists of constants $\texttt{Con}$ ($\texttt{Con} \subseteq \gV$), variables $\texttt{Var}$ ($\texttt{Var} \subseteq \gV$, existentially quantified), relation \emph{projections} $R(a, b)$ denoting a binary function over constants or variables, and logic symbols ($\exists, \land, \lor, \lnot$).
The answers $A_{\gG}(q)$ to the query $q$ are assignments of variables in a formula such that the instantiated query formula is a subgraph of the complete, non observable graph $\hat{\gG}$.
Answers are denoted as \emph{easy} if they are reachable by graph traversal over the incomplete graph $\gG$ and denoted as \emph{hard} if at least one edge from the non observable, complete graph $\hat{\gG}$ has to be predicted during query execution.

For example, in \autoref{fig:intro}, a query \emph{Which band member of Dire Straits played guitar?} is expressed in the logical form as $?U:\texttt{BandMember}(\texttt{Dire Straits}, U) \wedge \texttt{Plays}^{-1}(\texttt{Guitar}, U)$ as an \emph{intersection} query. 
Here, $U$ is a projected target variable, \texttt{Dire Straits} and \texttt{Guitar} are constants, \texttt{BandMember} and \texttt{Plays} are \emph{relation projections} where $\texttt{Plays}^{-1}$ denotes the inverse of the relation \texttt{Plays}.
The task of \clqa~ is to predict bindings (mappings between entities and variables) of the target variable, \eg, for the example query the answer set is a single entity $\gA_{q} = \{(U, \texttt{Mark Knopfler})\}$ and we treat this answer as an \emph{easy} answer as it is reachable by traversing the edges of the given graph.
In practice, however, we measure the performance of \clqa approaches on \emph{hard} answers.

\textbf{Inductive Query Answering.}
In the transductive \clqa setup, the training and inference graphs are the same and share the same set of entities and relations, \ie, $\gtrain = \ginf$ meaning that inference queries operate on the same graph, the same set of constants \texttt{Con} and relations.
This allows query answering models to learn hardcoded entity and relation embeddings at the same time losing the capabilities to generalize to new graphs at test time.

In the inductive entity $(e)$ setup considered in \citet{galkin2022}, the inference graph extends the training graph $\gtrain \subset \ginf$ but the set of relations is still fixed. Therefore, the proposed models are still bound to a certain hardcoded set of relations and cannot generalize to any arbitrary KG.

In this work, we lift all the restrictions on the training and inference graphs' vocabularies and consider the most general, inductive $(e,r)$ case when $\ginf \neq \gtrain$ and the inference graph might contain a completely different set of entities and relation types.
Furthermore, missing links still have to be predicted in the inference graphs to reach \emph{hard} answers.
% For example, in \autoref{fig:intro}, $\gtrain$ contains relations $\{\texttt{win}, \texttt{citizen}, \texttt{graduate} \}$\footnote{And their respective inverses} describing academic entities whereas $\ginf$ contains relations $\{\texttt{band member}, \texttt{plays}\}$ describing music industry.

%\zz{Not sure whether it is necessary to mention labeling tricks.}
\textbf{Labeling Trick GNNs.}
Labeling tricks (as coined by \citet{labeling_trick}) are featurization strategies in graphs for breaking % neighborhood symmetries
symmetries in node representations which are particularly pronounced in link prediction and KG completion tasks. 
In the presence of such node symmetries (\emph{automorphisms}), classical uni- and multi-relational GNN encoders~\citep{gcn, gat, compgcn} assign different \emph{automorphic} nodes the same feature making them indistinguishable for downstream tasks.
In multi-relational graphs, NBFNet~\citep{nbfnet} and A*Net~\citep{astarnet} apply a labeling trick by using the indicator function $\textsc{Indicator}(h,v,r)$ that puts a query vector $\vr$ on a head node $h$ and puts the zeros vector on other nodes $v$. 
The indicator function does not require entity embeddings and such models can naturally generalize to new %, unseen 
entities (while the set of relation types is still fixed).
Theoretically, such a labeling strategy learns \emph{conditional node representations} and is provably more powerful~\citep{rwl2} than node-level GNN encoders. 
In \clqa, only GNN-QE~\citep{gnn_qe} applies NBFNet as a projection operator making it the only approach generalizable to the inductive $(e)$ setup~\citep{galkin2022} with new nodes at inference time. 
This work leverages labeling trick GNNs to generalize \clqa to arbitrary KGs with any entity and relation vocabulary.
\begin{figure*}[t]
    \centering
    \includegraphics[width=\linewidth]{figs/Fig2v2.pdf}
    \vskip -0.1 in
    \caption{\textbf{(a)} Example of \emph{ip} query answering with \method: the inductive parametric projection operator (\Cref{subsec:ultra_proj}) executes relation projections on any graph and returns a scalar score for each entity; the scores are aggregated by non-parametric logical operators (\Cref{subsec:logic_ops}) implemented with fuzzy logics. Intermediate scores are used for weighted initializion of relation projections on the next hop. \textbf{(b)} The multi-source propagation issue with a pre-trained link predictor for relation projection: pre-training on  \emph{1p} link prediction is done in the single-source labeling mode (top) where only one query node is labeled with a non-zero vector; complex queries at later intermediate hops might have several plausible sources with non-zero initial weights %from intermediate results 
    (bottom) where a pre-trained operator fails. }
    \label{fig:ultraquery}
    \vskip -0.15 in
\end{figure*}

\section{Method}
\label{sec:method}

We aim at designing a single foundation model for \clqa on any KG in the zero-shot fashion, \ie, without training on a target graph.
In the \clqa literature~\citep{gqe,q2b,betae,cqd,gnn_qe}, it is common to break down query execution into a \emph{relation projection} to traverse graph edges and predict missing links, and \emph{logical operators} that model conjunction, disjunction, and union.
The main challenge boils down to designing inductive projection and logical operators suitable for any % unseen
entity and relation vocabulary.

\subsection{Inductive Relation Projection}
\label{subsec:ultra_proj}

The vast majority of \clqa methods are inherently transductive and implement relation projections as functions over entity and relation embeddings fixed to a certain KG vocabulary, \eg, with scoring functions from KG completion methods~\citep{gqe,cqd,qto}, geometric functions~\citep{q2b,cone}, or pure neural methods~\citep{mlpmix,lmpnn}.
The only method inductive to new entities~\citep{gnn_qe} learns relation embeddings and uses those as a labeling trick (\Cref{sec:prelim}) for a GNN that implements the projection operator.

As fixed relation embeddings do not transfer to new KGs with new relations, 
%in \method, 
we adapt \ultra~\citep{ultra}, an inductive approach that builds relation representations dynamically using the invariance of \emph{relation interactions}, as the backbone of the relation projection operator thanks to its good zero-shot performance on simple KG completion tasks across a variety of graphs.
\ultra leverages theoretical findings in multi-relational link prediction~\citep{rwl,rwl2} and learns relation representations from a \emph{meta-graph} of relation interactions\footnote{The meta-graph can be efficiently obtained from any KG.}.
The meta-graph includes four learnable edge types or meta-relations (\emph{head-to-tail}, \emph{tail-to-head}, \emph{head-to-head}, \emph{tail-to-tail}) which are independent from KG's relation vocabulary and therefore transfer across any graph.
Practically, given a graph $\gG$ and projection query $(h, r, ?)$, \ultra employs labeling trick GNNs on two levels.
First, it builds a meta-graph $\gG_r$ of relation interactions (a graph of relations where each node is a unique edge type in $\gG$) and applies a labeling trick to initialize the query node $r$. 
Note that $|\gR| \ll |\gE|$, the number of unique relations is much smaller than number of entities in any KG, so processing this graph of relations introduces a rather marginal computational overhead.
Running a message passing GNN over $\gG_r$ results in \emph{conditional relation representation} which are used as initial edge type features in the second, entity-level GNN.
There, a starting node $h$ is initialized with a query vector from the obtained relation representations and running another GNN over the entity graph (with a final sigmoid readout) returns a scalar score in $[0, 1]$ representing a probability of each node to be a tail of a query $(h,r,?)$.

The only learnable parameters in \ultra are four meta-relations for the graph of relations and GNN weights. The four meta-relations represent structural patterns 
%like “a tail of and edge with relation X is the head of another edge with relation Y” 
and can be mined from any multi-relational KG independent of their entity and relation vocabulary. 
GNN weights are optimized during pre-training. Since the model does not rely on any KG-specific entity or relation vocabulary, a single pre-trained \ultra model can be used as a zero-shot relation projection operator on any KG. \autoref{fig:ultraquery}(a) illustrates the \emph{intersection-projection} query execution process where each projection step is tackled by the same inductive projection operator with initialization depending on the start anchor node or intermediate variables.

\textbf{The multi-source propagation issue.}
% \zz{We haven't introduced any idea of pre-trained KG models for relation projection so far.}
While it is tempting to leverage \ultra pre-trained on multiple KG datasets for relation projection, there is a substantial distribution shift (\autoref{fig:ultraquery}(b)) between KG completion and \clqa. Specifically, KG completion is a special case of relation projection where the input always contains a single node.
% The intermediate steps, however, present a conceptual challenge (\autoref{fig:ultraquery}(b)), that is, 
% a relation projection operator pre-trained on KG completion (equivalently, on \emph{1p} queries) is tailored for the \emph{single source, multiple tails} message passing when only one query node is initialized with a non-zero vector. 
% On the other hand,
By comparison, in multi-hop complex queries, several likely nodes might have high intermediate scores and will be labeled with non-zero vectors leading to the \emph{multiple sources} %, multiple tails}
propagation mode where a pre-trained operator is likely to fail.
To alleviate the issue, we experimentally study two strategies: (1) short fine-tuning  of the pre-trained projection operator on complex queries (used in the main \method model), or (2) use the frozen pre-trained operator and threshold intermediate scores setting all scores below $0 < k < 1$ to zero (denoted as \methodlp). The insight is to limit the propagation to one or a few source nodes, thereby reducing the discrepancy between training and test distributions.

%\ultra demonstrates generalization to dozens of unseen KGs in the zero-shot fashion outperforming many tailored inductive baselines.

% \zz{May retell Vapnik's definition of transductive and inductive inference here.
% \url{https://axon.cs.byu.edu/~martinez/classes/778/Papers/transductive.pdf}}

% \zz{
% \begin{enumerate}
%     \item (4.1) Definition of inductive and transductive models. May adapt from Vapnik's definition. Generally the idea is that transductive setting assumes the test vocabulary is known beforehand (which is identical to the training one in the setup of KGs).
%     \item Our goal: develop an inductive model for CLQA.
%     \item Following existing works (GQE, Q2B, BetaE, CQD, GNN-QE, etc.), CLQA is often handled by decomposing the query into relation projection operation and logical operations (conjunction, disjunction, negation).
%     \item The challenge boils down to: how to design inductive relation projection operations, and inductive logical operations? We try to answer this question with designs from existing methods.
%     \item (4.2) Inductive relation projection: Discuss why GQE, Q2B, BetaE, CQD are not inductive. Discuss why GNN-QE is inductive for entity vocabularies, but not for relation vocabularies.
%     \item How can we make a relational GNN (i.e. the family of suitable GNNs for GNN-QE) inductive for relation vocabularies? The idea is to parameterize relation embeddings as a function of a meta graph describing the relations (cite Ultra).
% \end{enumerate}
% }

% \zz{The implication of this story: based on what we know about existing complex query methods (we know a lot in our survey), the simplest choice for an inductive model is to combine GNN projection, relation meta-graph and fuzzy logic operation. Try not to mention GNN-QE as a whole, but derive GNN projection and fuzzy logic operation separately.}
% \subsection{GNN-QE as Inductive Query Executor}
%\paragraph{Learning Relation Representations from a Meta Graph}
%\zz{Reviewers didn't know NBFNet when we wrote GNN-QE. I would not assume the reviewers know Ultra well. So try to retell the idea and cite the paper rather than refer to the method.}
% \subsection{\ultra as a Universal Relation Projection Module}


\subsection{Inductive Logical Operations}
\label{subsec:logic_ops}

%     \item (4.3) Inductive logical operations: Discuss why GQE, Q2B, BetaE are not inductive. Their operations are learned to fit a particular entity embedding space. The only inductive solution from existing method is the t-norms and t-conorms from CQD, GNN-QE, QTO.
%     \item (4.4) Training \& optimization tricks: e.g. thresholding. Or maybe we want to put them to implementation details?
Learnable logical operators parameterized by neural nets in many \clqa approaches~\cite{gqe,q2b,cone,mlpmix} fit a particular embedding space and are not transferable.
Instead, we resort to differentiable but non-parametric \emph{fuzzy logics}~\citep{vankrieken_fuzzy} that implement logical operators as algebraic operations (\emph{t-norms} for conjunction and \emph{t-conorms} for disjunction) in a bounded space $[0,1]$ and are used in several neuro-symbolic \clqa approaches~\citep{cqd, gnn_qe, cqda, qto, fit}. 
%In fuzzy logics, \emph{t-norms} correspond to soft conjunctions, \emph{t-conorms} to disjunctions, and negation is often implemented as $1-x$ for bounded
\method employs fuzzy logical operators over \emph{fuzzy sets} $\vx \in [0,1]^{|\gV|}$ as the relation projection operator assigns a scalar in range $[0,1]$ for each entity in a graph. 
The choice of a fuzzy logic is often a hyperparameter although \citet{vankrieken_fuzzy} show that the \emph{product logic} is the most stable. 
In product logic, given two fuzzy sets $\vx, \vy$, conjunction is element-wise multiplication $\vx \odot \vy$ and disjunction is $\vx + \vy - \vx \odot \vy$.
Negation is often implemented as $\mathbf{1}-\vx$ where $\mathbf{1}$ is the \emph{universe} vector of all ones.
For second- and later $i$-th hop projections, we obtain initial node states $\vh_v$ by weighting a query vector $\vr_i$ with their probability score $x_v$ from the fuzzy set of a previous step: $\vh_v = x_v \vr_i$.


\subsection{Training}
Following existing works~\citep{betae,gnn_qe}, \method is trained on complex queries to minimize the binary cross entropy loss
\begin{align}
%\resizebox{.88\columnwidth}{!}{
\hspace{-0.5em}\displaystyle{
        \gL = -\frac{1}{|\gA_{q}|}\sum_{a \in \gA_{q}}\log p(a|q) 
              -\frac{1}{|\gV\backslash\gA_{q}|}\sum_{a' \in \gV\backslash\gA_{q}}\log (1 - p(a'|q))}
              %}
    \label{eqn:loss}
\end{align}
where $\gA_{q}$ is the answer to the query $q$ and $p(a|q)$ is the probability of entity $a$ in the final output fuzzy set.
\methodlp uses a frozen checkpoint from KG completion and is not trained on complex logical queries.

% talk about the main challenge: single-source message propagation vs multi-source propagation in the later hops of query answering
\section{Experiments}
\label{sec:experiments}

% Our evaluation agenda for \method includes the 
Our experiments focus on the following research questions: (1) How does a single \method model perform in the zero-shot inference mode on unseen graphs and queries compared to the baselines? (2) Does \method retain the quality metrics like \emph{faithfullness} and identify easy answers reachable by traversal? (3) How does the multi-source propagation issue affect the performance?

\subsection{Setup and Datasets}

\textbf{Datasets.} 
We employ 23 different \clqa datasets each with 14 standard query types and its own underlying KG with different sets of entities and relations. Following Section~\ref{sec:prelim}, we categorize the datasets into three groups (more statistics of the datasets and queries are provided in Appendix~\ref{app:datasets}):
\begin{itemize}[leftmargin=*]
    \item \emph{Transductive} (3 datasets) where training and inference graphs are the same $(\gtrain = \ginf)$ and test queries cover the same set of entities and relations: FB15k237, NELL995 and FB15k all from \citet{betae} with at most 100 answers per query.
    \item \emph{Inductive entity} $(e)$ (9 datasets) from \citet{galkin2022} where inference graphs extend training graphs $(\gtrain \subset \ginf)$ being up to 550\% larger in the number of entities. The set of relations is fixed in each training graph and does not change at inference making the setup inductive with respect to the entities. Training queries might have more true answers in the extended inference graph.
    \item \emph{Inductive entity and relation} $(e,r)$ (11 datasets): we sampled a novel suite of WikiTopics-QA datasets due to the absence of standard benchmarks evaluating the hardest inductive setup where inference graphs have both new entities and relations $(\gtrain \neq \ginf)$. The source graphs were adopted from the WikiTopics datasets~\citep{isdea}, we follow the \emph{BetaE setting} when sampling 14 query types with at most 100 answers. More details on the dataset creation procedure are in Appendix~\ref{app:datasets}. 
\end{itemize} 

\textbf{Implementation and Training.}
\method was trained on one FB15k237 dataset with complex queries for 10,000 steps with batch size of 32 on 4 RTX 3090 GPUs for 2 hours (8 GPU-hours in total). 
We initialize the model weights with an available checkpoint of \ultra reported in \citet{ultra}. 
Following the standard setup in the literature, we train the model on 10 query types and evaluate on all 14 patterns.
We employ \emph{product t-norm} and \emph{t-conorm} as non-parametric fuzzy logic operators to implement conjunction $(\wedge)$ and disjunction $(\lor)$, respectively, and use a simple $1-x$ negation.
For the ablation study, \methodlp uses the same frozen checkpoint (pre-trained on simple \emph{1p} link prediction) with scores thresholding to alleviate the multi-source propagation issue (\Cref{subsec:ultra_proj}).
More details on all hyperparameters are available in Appendix~\ref{app:hyperparams}.



\textbf{Evaluation Protocol.}
As we train an \method model only on one FB15k237 dataset and run zero-shot inference on other 22 graphs, the inference mode on those is \emph{inductive} $(e,r)$ since their entity and relation vocabularies are all different from the training set.

As common in the literature~\citep{betae,ren2023ngdb}, the answer set of each query is split into \emph{easy} and \emph{hard} answers. 
Easy answers are reachable by graph traversal and do not require inferring missing links whereas hard answers are those that involve at least one edge to be predicted at inference.
In the rank-based evaluation, we only consider ranks of \emph{hard} answers and filter out easy ones and report filtered Mean Reciprocal Rank (MRR) and Hits@10 as main performance metrics.

Other qualitative metrics include: (1) \emph{faithfullness}~\citep{emql}, \ie, the ability to recover \emph{easy} answers reachable by graph traversal. 
Here, we follow the setup in \citet{galkin2022} and measure the performance of training queries on larger inference graphs where the same queries might have new true answers;
(2) the ROC AUC score to estimate whether a model ranks easy answers higher than hard answers -- we compute ROC AUC over \emph{unfiltered} scores of easy answers as positive labels and hard answers as negative.
(3) Mean Absolute Percentage Error (MAPE)~\citep{gnn_qe} between the number of answers extracted from model's predictions and the number of ground truth answers (easy and hard combined) to estimate whether \clqa models can predict the cardinality of the answer set.


\textbf{Baselines.}
In transductive and inductive $(e)$ datasets, we compare a single \method model with the best reported models trained end-to-end on each graph (denoted as \emph{Best baseline} in the experiments): QTO~\citep{qto} for 3 transductive datasets (FB15k237, FB15k, and NELL995) and GNN-QE~\citep{galkin2022} for 9 inductive $(e)$ datasets.
While a single \method model has 177k parameters, the baselines are several orders of magnitude larger with a parameters count depending on the number of entities and relations, \eg, a QTO model on FB15k237 has 30M parameters due to having 2000$d$ entity and relation embeddings, and GNN-QE on a reference FB 175\% inductive $(e)$ dataset has 2M parameters.
For a newly sampled suite of 11 inductive $(e, r)$ datasets, we compare against the edge-type heuristic baseline introduced in \citet{galkin2022}. 
The heuristic selects the candidate nodes with the same incoming relation as the last hop of the query. 
More details on the baselines are reported in Appendix~\ref{app:hyperparams}

\subsection{Main Experiment: Zero-shot Query Answering}

\begin{table*}[t]
\centering
\caption{Zero-shot inference results of \method and ablated \methodlp on 23 datasets compared to the best reported baselines. \method was trained on one transductive FB15k237 dataset, \methodlp was only pre-trained on KG completion and uses scores thresholding. The \emph{no thrs.} version does not use any thresholding of intermediate scores (\Cref{subsec:ultra_proj}). The best baselines are trainable on each transductive and inductive $(e)$ dataset, and the non-parametric heuristic baseline on inductive $(e,r)$ datasets. }
%\scriptsize
\begin{adjustbox}{width=\textwidth}
\begin{tabular}{lrrrrrrrrrrrrrrrrr}\toprule
\multirow{3}{*}{Model} &\multicolumn{4}{c}{\textbf{Inductive} $(e,r)$ (11 datasets)} &\multicolumn{4}{c}{\textbf{Inductive} $(e)$ (9 datasets)} &\multicolumn{4}{c}{\textbf{Transductive} (3 datasets)} &\multicolumn{4}{c}{\textbf{Total Average} (23 datasets)} \\\cmidrule(l){2-5} \cmidrule(l){6-9} \cmidrule(l){10-13} \cmidrule(l){14-17}
&\multicolumn{2}{c}{EPFO avg} &\multicolumn{2}{c}{neg avg} &\multicolumn{2}{c}{EPFO avg} &\multicolumn{2}{c}{neg avg} &\multicolumn{2}{c}{EPFO avg} &\multicolumn{2}{c}{neg avg} &\multicolumn{2}{c}{EPFO avg} &\multicolumn{2}{c}{neg avg} \\\cmidrule(l){2-3} \cmidrule(l){4-5} \cmidrule(l){6-7} \cmidrule(l){8-9} \cmidrule(l){10-11} \cmidrule(l){12-13} \cmidrule(l){14-15} \cmidrule(l){16-17}
&\bf{MRR} &\bf{H@10} &\bf{MRR} &\bf{H@10} &\bf{MRR} &\bf{H@10} &\bf{MRR} &\bf{H@10} &\bf{MRR} &\bf{H@10} &\bf{MRR} &\bf{H@10} &\bf{MRR} &\bf{H@10} &\bf{MRR} &\bf{H@10} \\\midrule
Best baseline &0.014 &0.029 &0.004 &0.007 &\bf{0.328} &\textbf{0.469} &\bf{0.176} &\bf{0.297} &\bf{0.468} &\bf{0.603} &\bf{0.259} &\bf{0.409} &0.196 &0.276 &0.105 &0.173 \\ \midrule
\textsc{UltraQuery} 0-shot &\bf{0.280} &0.380 &\bf{0.193} &\bf{0.288} &0.312 &\bf{0.467} &0.139 &0.262 &0.411 &0.517 &0.240 &0.352 &\bf{0.309} &\bf{0.432} &\bf{0.178} &\bf{0.286} \\
\textsc{UltraQuery LP} 0-shot &0.268 &\bf{0.409} &0.104 &0.181 &0.277 &0.441 &0.098 &0.191 &0.322 &0.476 &0.150 &0.263 &0.279 &0.430 &0.107 &0.195 \\
\textsc{UltraQuery LP} no thrs. &0.227 &0.331 &0.080 &0.138 &0.246 &0.390 &0.085 &0.167 &0.281 &0.417 &0.127 &0.223 &0.242 &0.367 &0.088 &0.161 \\
\bottomrule
\end{tabular}
\end{adjustbox}
\label{tab:maintab1}
\vskip -0.1 in
\end{table*}
\begin{figure*}[t]
    \centering
    \includegraphics[width=\linewidth]{figs/newres_pyg.pdf}
    \vskip -0.1 in
    \caption{Mitigation of the multi-source message passing issue (Section~\ref{sec:method}) with \method: while \methodlp (pre-trained only on 1p link prediction) does reach higher 1p query performance (center right), it underperforms on negation queries (center left). \method adapts to the multi-source message passing scheme and trades a fraction of 1p query performance for better averaged EPFO, \eg, on the \emph{3i} query (right), and negation queries performance. More results are in Appendix~\ref{app:more_results}.}
    \label{fig:abl_multisource}
    \vskip -0.1 in
\end{figure*}

In the main experiment, we measure the zero-shot query answering performance of \method trained on a fraction of complex queries of one FB15k237 dataset. 
Figure~\ref{fig:main_fig1} and Table~\ref{tab:maintab1} illustrate the comparison with the best available baselines and ablated \methodlp model on 23 datasets split into three categories (transductive, inductive $(e)$, and inductive $(e,r)$). 
For each dataset, we measure the average MRR on 9 EPFO queries with projection, intersection, and union operators, and 5 negation queries with the negation operator, respectively.

Averaged across 23 datasets, \method outperforms available baselines by relative 50\% in terms of MRR and Hits@10 on EPFO and 70\% on negation queries (\eg, 0.31 vs 0.20 MRR on EPFO queries and 0.178 vs 0.105 on negation queries). 
The largest gains are achieved on the hardest inductive $(e,r)$ datasets where the heuristic baseline is not able to cope with the task. 
On inductive $(e)$ datasets, \method outperforms the trainable SOTA GNN-QE model on larger inductive inference graphs and performs competitively on smaller inductive versions. 
On transductive benchmarks, \method lags behind the SOTA QTO model which is expected and can be attributed to the sheer model size difference (177k of \method vs 30M of QTO) and the computationally expensive brute-force approach of QTO that materializes the whole $(\gV \times \gV \times \gR)$ 3D tensor of scores of all possible triples. 
Pre-computing such tensors on three datasets takes considerable space and time, \eg, 8 hours for FB15k with heavy sparsification settings to fit onto a 24 GB GPU.
%Besides, the FB15k dataset is known for years to have leakages~\citep{fb15k237} from inverse edges that large embedding models memorize and exploit. 
Still, \method outperforms a much larger QTO model on the FB15k dataset on both EPFO and negation queries.
The graph behind the NELL995 dataset is a collection of disconnected components which is disadvantageous for GNNs.

We note a decent performance of \methodlp trained only on simple \emph{1p} link prediction and imbued with score thresholding to alleviate the multi-source message passing issue described in Section~\ref{subsec:ultra_proj}.
Having a deeper look at other qualitative metrics in the following section, we reveal more sites where the issue incurs negative effects.


\subsection{Analysis}


\begin{figure*}[t]
    \centering
    \includegraphics[width=\linewidth]{figs/qual_analysis_pyg.pdf}
    \vskip -0.1 in
    \caption{Qualitative analysis on 9 inductive $(e)$ and 3 transductive datasets averaged across all 14 query types. \textbf{Faithfullness, MRR (left):} \method successfully finds easy answers in larger inference graphs and outperforms trained GNN-QE baselines. \textbf{Ranking of easy vs hard answers, ROC AUC (center):} zero-shot inference methods slightly lag behind trainable GNN-QE due to assigning higher scores to hard answers. \textbf{Cardinality Prediction, MAPE (right):} \method is comparable to a much larger trainable baseline QTO. In all cases, \methodlp is significantly inferior to the main model. }
    \label{fig:abl_quality}
    \vskip -0.2 in
\end{figure*}

Here, we study four aspects of model performance: the effect of the multi-source message passing issue mentioned in Section~\ref{subsec:ultra_proj}, the ability to recover answers achievable by edge traversal (\emph{faithfullness}), the ability to rank easy answers higher than hard answers, and the ability to estimate the cardinality of the answer set.

\textbf{The multi-source message passing effect.}
%\zz{Maybe it's better to add results of \methodlp without thresholding.}
The pre-trained \ultra checkpoint used in \methodlp is tailored for singe-source message passing and struggles in the \clqa setup on later hops with several initialized nodes (\autoref{tab:maintab1}). 
Training \method on complex queries alleviates this issue as shown in \autoref{fig:abl_multisource}, \ie, while \emph{1p} performance of \methodlp is higher, the overall performance on EPFO and negative queries is lacking. 
In contrast, \method trades a fraction of \emph{1p} single-source performance to a much better performance on negative queries (about $2\times$ improvement) and better performance on many EPFO queries, for example, on \emph{3i} queries. 
Besides that, we note that the zero-shot performance of both \method models does not deteriorate from the increased size of the inference graph compared to the baseline GNN-QE.

\textbf{Recovering easy answers on any graph.}
\emph{Faithfullness}~\citep{emql} is the ability of a \clqa model to return \emph{easy} query answers, \ie, the answers reachable by edge traversal in the graph without predicting missing edges.
While faithfullness is a common problem for many \clqa models, \autoref{fig:abl_quality} demonstrates that \method almost perfectly recovers easy answers on any graph size even in the zero-shot inference regime in contrast to the best baseline.
Simple score thresholding does not help \methodlp to deal with complex queries as all easy intermediate nodes have high scores above the threshold and the multi-source is more pronounced. 

\textbf{Ranking easy and hard answers.}
A reasonable \clqa model is likely to score easy answers higher than hard ones that require inferring missing links~\citep{galkin2022}.
Measuring that with ROC AUC  (\autoref{fig:abl_quality}), \method is behind the baseline due to less pronounced decision boundaries (overlapping distributions of scores) between easy and hard answers' scores. 
Still, due to scores filtering when computing ranking metrics, this fact does not have a direct negative impact on the overall performance.

\textbf{Estimating the answer set cardinality.}
Neural-symbolic models like GNN-QE and QTO have the advantage of estimating the cardinality of the answer set based on the final scores without additional supervision. 
As shown in \autoref{fig:abl_quality}, \method is comparable to the larger and trainable QTO baseline on FB15k237 (on which the model was trained) as well as on other datasets in the zero-shot inference regime. 
Since cardinality estimation is based on score thresholding, \methodlp is susceptible to the multi-source propagation issue with many nodes having a high score and is not able to deliver a comparable performance.


\begin{wrapfigure}{R}{0.5\textwidth}
\begin{minipage}{0.5\textwidth}
\vspace{-3em}
    \centering
    \includegraphics[width=1.0\linewidth]{figs/ultraquery_num_graphs.pdf}
    \caption{Average MRR (left) and Hits@10 (right) of 9 inductive $(e)$ and 11 inductive $(e,r)$ \clqa datasets for EPFO and negation queries depending on the number of graphs in the training mix.}
    \label{fig:num_graphs1}
    \vspace{-1em}
\end{minipage}
\end{wrapfigure}


\textbf{Varying the number of graphs in training.}
\autoref{fig:num_graphs1} and \autoref{tab:rebuttal2} report the inductive inference \clqa performance depending on the number of KGs in the training mixture. 
The original \method was trained on queries from the FB15k237. 
In order to maintain the zero-shot inductive inference setup on 11 inductive $(e,r)$ and 9 inductive $(e)$ datasets, we trained new model versions on the rest of BetaE datasets, that is, \method \textsc{2G} combines FB15k237 and NELL995 queries (trained for 20k steps), \method \textsc{3G} combines FB15k273, NELL995, and FB15k queries (trained for 30k steps). %The training parameters are the same as in the Appendix B and Table 9 from the original paper except that the \method \textsc{2G} was trained for 20k steps and \method \textsc{3G} was trained on 30k steps (reported in Table~\ref{tab:rebuttal1}).
The most noticeable improvement of \textsc{2G} and \textsc{3G} versions is the increased MRR and Hits@10 on EPFO queries (9 query types) on 11 inductive $(e,r)$  datasets yielding about 10\% gains.
On 9 inductive $(e)$ datasets the performance is either on par with the \textsc{1G} version or a bit lower. 
Averaged across 20 datasets, the \textsc{2G} version exhibits the best EPFO performance at the cost of slightly reduced negation query performance.




\begin{table}[t]
\centering
\caption{Zero-shot inference results (on 20 inductive datasets) of \method trained on 1, 2, and 3 datasets, respectively. The biggest gains of the 2G model are \textbf{in bold}.}
\label{tab:rebuttal2}
%\scriptsize
\begin{adjustbox}{max width=\textwidth}
\begin{tabular}{lrrrrrrrrrrrrr}\toprule
\multirow{3}{*}{Model} &\multicolumn{4}{c}{\textbf{Inductive} $(e,r)$ (11 datasets)} &\multicolumn{4}{c}{\textbf{Inductive} $(e)$ (9 datasets)} &\multicolumn{4}{c}{\textbf{Total Average} (20 datasets)} \\\cmidrule{2-13}
&\multicolumn{2}{c}{EPFO avg} &\multicolumn{2}{c}{neg avg} &\multicolumn{2}{c}{EPFO avg} &\multicolumn{2}{c}{neg avg} &\multicolumn{2}{c}{EPFO avg} &\multicolumn{2}{c}{neg avg} \\\cmidrule{2-13}
&\bf{MRR} & \bf{H@10} & \bf{MRR} &\bf{H@10} &\bf{MRR} &\bf{H@10} &\bf{MRR} &\bf{H@10} &\bf{MRR} &\bf{H@10} &\bf{MRR} &\bf{H@10} \\\midrule
\method 1G &0.280 &0.380 &0.193 &0.288 &0.312 &0.467 &0.139 &0.262 &0.296 &0.423 &0.166 &0.275 \\
\method 2G & \bf{0.310} & \bf{0.413} &0.187 &0.275 &0.307 &0.463 &0.130 &0.244 &0.308 &0.438 &0.158 &0.260 \\
\method 3G &0.304 &0.402 &0.195 &0.292 &0.292 &0.438 &0.127 &0.239 &0.298 &0.420 &0.161 &0.265 \\
\bottomrule
\end{tabular}
\end{adjustbox}
\vspace{-1em}
\end{table}
\section{Discussion and Future Work}
\label{sec:conclusion}

\textbf{Limitations.}
The only parameterized component of \method is the projection operator and, therefore, limitations and improvement opportunities stem from the projection operator~\citep{ultra} and its interplay with the multi-hop query answering framework. 
For instance, new mechanisms of tackling the multi-source propagation, better pre-training strategies, and scaling might positively impact the zero-shot \clqa performance.
The support for very large KGs could be further improved by adopting more scalable entity-level GNN predictors like A*Net~\citep{astarnet} or AdaProp~\citep{adaprop} which have been shown to scale to graphs of millions of nodes. We are optimistic that \method could scale to such graphs  when integrated with those models.

\textbf{Conclusion and Future Work.}
We presented \method, the first foundation model for inductive zero-shot complex logical query answering on any KG that combines a parameterized, inductive projection operator with non-parametric logical operators.
Alleviating the multi-source message propagation issue is the key to adapt pre-trained projection operators into the multi-hop query answering framework.
\method performs comparably to or better than strong baselines trained specifically on each graph and at the same time retains key qualitative features like faithfullness and answer cardinality estimation.
Having a single query answering model working on any KG, the scope for future work is vast as highlighted by \citet{ren2023ngdb} and includes, for example, better theoretical understanding of logical expressiveness bounds, supporting more query patterns beyond simple trees~\citep{efok, fit}, queries without anchor nodes~\citep{egnn_qe}, hyper-relational queries~\citep{starqe}, queries with numerical literals~\citep{demir_litcqd}, or temporal queries~\citep{lin2023tflex}.

\textbf{Impact Statement.}
We do not envision direct ethical or societal consequences of this work. 
Still, models capable of zero-shot inference on any graph might be applied to domains other than those designed by the authors. 
Positive impacts include saving compute resources and reducing carbon footprint of training specific models tailored for each graph.

\vspace{-1em}
\section*{Acknowledgements}

This project is supported by Intel-Mila partnership program, the Natural Sciences and Engineering Research Council (NSERC) Discovery Grant, the Canada CIFAR AI Chair Program, collaboration grants between Microsoft Research and Mila, Samsung Electronics Co., Ltd., Amazon Faculty Research Award, Tencent AI Lab Rhino-Bird Gift Fund and a NRC Collaborative R\&D Project (AI4D-CORE-06). This project was also partially funded by IVADO Fundamental Research Project grant PRF-2019-3583139727. The computation resource of this project is supported by Mila\footnote{\url{https://mila.quebec/}}, Calcul Qu\'ebec\footnote{\url{https://www.calculquebec.ca/}} and the Digital Research Alliance of Canada\footnote{\url{https://alliancecan.ca/}}.

This work was funded in part by the National Science Foundation (NSF) awards, CCF-1918483, CAREER IIS-1943364 and CNS-2212160, Amazon Research Award, AnalytiXIN, and the Wabash Heartland Innovation Network (WHIN). Computing infrastructure was supported in part by CNS-1925001 (CloudBank). Any opinions, findings and conclusions or recommendations expressed in this material are those of the authors and do not necessarily reflect the views of the sponsors.

% \begin{figure}[ht]
% \vskip 0.2in
% \begin{center}
% \centerline{\includegraphics[width=\columnwidth]{icml_numpapers}}
% \caption{Historical locations and number of accepted papers for International
% Machine Learning Conferences (ICML 1993 -- ICML 2008) and International
% Workshops on Machine Learning (ML 1988 -- ML 1992). At the time this figure was
% produced, the number of accepted papers for ICML 2008 was unknown and instead
% estimated.}
% \label{icml-historical}
% \end{center}
% \vskip -0.2in
% \end{figure}


\bibliography{bibliography}
\bibliographystyle{plainnat}


%%%%%%%%%%%%%%%%%%%%%%%%%%%%%%%%%%%%%%%%%%%%%%%%%%%%%%%%%%%%%%%%%%%%%%%%%%%%%%%
%%%%%%%%%%%%%%%%%%%%%%%%%%%%%%%%%%%%%%%%%%%%%%%%%%%%%%%%%%%%%%%%%%%%%%%%%%%%%%%
% APPENDIX
%%%%%%%%%%%%%%%%%%%%%%%%%%%%%%%%%%%%%%%%%%%%%%%%%%%%%%%%%%%%%%%%%%%%%%%%%%%%%%%
%%%%%%%%%%%%%%%%%%%%%%%%%%%%%%%%%%%%%%%%%%%%%%%%%%%%%%%%%%%%%%%%%%%%%%%%%%%%%%%
\newpage
\appendix
\onecolumn

\section{Datasets}
\label{app:datasets}

\subsection{Dataset statistics}
First, \autoref{tab:app_datasets_transd}, \autoref{tab:app_datasets_inde}, and \autoref{tab:app_datasets_indr} provide the necessary details on the graphs behind the \clqa datasets. 
Then, \autoref{tab:app_transd_queries}, \autoref{tab:app_inductive_e_queries}, and \autoref{tab:wikitopics-clqa} list the query statistics. 
Transductive datasets are BetaE~\citep{betae} datasets (MIT license), inductive $(e)$ datasets are adopted from \citet{galkin2022} (CC BY 4.0 license) where validation and test inference graphs extend the training graph. The ratio denotes the size of the inference graph to the size of the training graph (in the number of nodes), that is, $\gV_{\textit{inf}} / \gV_{\textit{train}}$.
In the following \Cref{app:subsec_wikitopcs_clqa} we provide more details in sampling 11 new inductive $(e,r)$ datasets WikiTopics-CLQA (available under the CC BY 4.0 license).


\begin{table}[!ht]
\centering
\caption{Graph in transductive datasets (3) from \citet{betae}. Inverse triples and edge types are included in the splits. Train, Valid, Test denote triples in the respective set. }
\label{tab:app_datasets_transd}
%\scriptsize
%\begin{adjustbox}{width=\textwidth}
\begin{tabular}{lccccc}\toprule
Dataset &Entities &Rels &Train &Valid &Test  \\\midrule
FB15k & 14,951 & 2,690 & 966,284 & 100,000 & 118,142 \\
FB15k237 &14,505 &474 &544,230 &35,052 &40,876 \\
NELL995  &63,361 &400 &228,426 &28,648 & 28,534 \\
\bottomrule
\end{tabular}
%\end{adjustbox}
%
\caption{Graphs in inductive $(e)$ datasets (9) from \citet{galkin2022}. Inverse triples and edge types are included in all graphs. Validation and Test splits contain an inference graph $(\gV_\textit{inf}, \gE_\textit{inf})$ which is a superset of the training graph with new nodes, and missing edges to predict (Valid and Test, respectively).  }
\label{tab:app_datasets_inde}
%\scriptsize
\begin{adjustbox}{width=\textwidth}
\begin{tabular}{ccccccccccc}\toprule
\multirow{2}{*}{Ratio, \%} &\multirow{2}{*}{Rels} &\multicolumn{2}{c}{Training Graph} &\multicolumn{3}{c}{Validation Graph} &\multicolumn{3}{c}{Test Graph} \\ \cmidrule(lr){3-4}  \cmidrule(lr){5-7}  \cmidrule(lr){8-10}
& &Entities & Triples & Entities & Triples & Valid & Entities & Triples & Test \\ \midrule
106\% &466 &13,091 &493,425 &13,801 &551,336 &10,219 &13,802 &538,896 &8,023 \\
113\% &468 &11,601 &401,677 &13,022 &491,518 &15,849 &13,021 &486,068 &14,893 \\
122\% &466 &10,184 &298,879 &12,314 &413,554 &20,231 &12,314 &430,892 &23,289 \\
134\% &466 &8,634 &228,729 &11,468 &373,262 &25,477 &11,471 &367,810 &24,529 \\
150\% &462 &7,232 &162,683 &10,783 &311,462 &26,235 &10,782 &331,352 &29,755 \\
175\% &436  &5,560 &102,521 &9,801 &265,412 &28,691 &9,781 &266,494 &28,891 \\
217\% &446  &4,134 &52,455 &9,062 &227,284 &30,809 &9,058 &212,386 &28,177 \\
300\% &412  &2,650 &24,439 &8,252 &178,680 &27,135 &8,266 &187,156 &28,657 \\
550\% &312  &1,084 &5,265 &7,247 &136,558 &22,981 &7,275 &133,524 &22,503 \\
\bottomrule
\end{tabular}
\end{adjustbox}
\caption{Graphs in the newly sampled inductive entity and relation $(e,r)$ WikiTopics-CLQA datasets (11). Triples denote the number of edges of the graph given at training, validation, or test. Valid and Test denote triples to be predicted in the validation and test sets in the respective validation and test graph. 
% In accordance with the setting of WikiTopics~\cite{isdea}, the validation graph is a superset of the training graph, but the test graph is a 
}
\label{tab:app_datasets_indr}
%\scriptsize
\begin{adjustbox}{width=\textwidth}
\begin{tabular}{lcccccccccccc}\toprule
\multirow{2}{*}{Dataset} &\multicolumn{3}{c}{Training Graph} &\multicolumn{4}{c}{Validation Graph} &\multicolumn{4}{c}{Test Graph} \\ \cmidrule(l){2-4} \cmidrule(l){5-8} \cmidrule(l){9-12}
&Entities &Rels &Triples &Entities &Rels &Triples &Valid &Entities &Rels &Triples &Test \\\midrule
Art &10000 &65 &27262 &10000 &65 &27262 &3026 &10000 &65 &28023 &3113 \\
Award &10000 &17 &23821 &10000 &13 &23821 &2646 &10000 &17 &25056 &2783 \\
Education &10000 &19 &14355 &10000 &19 &14355 &1594 &10000 &19 &14193 &1575 \\
Health &10000 &31 &15539 &10000 &31 &15539 &1725 &10000 &31 &15337 &1703 \\
Infrastructure &10000 &37 &21990 &10000 &37 &21990 &2443 &10000 &37 &21646 &2405 \\
Location &10000 &62 &85063 &10000 &62 &85063 &9451 &10000 &62 &80269 &8917 \\
Organization &10000 &34 &33325 &10000 &34 &33325 &3702 &10000 &34 &31314 &3357 \\
People &10000 &40 &55698 &10000 &40 &55698 &6188 &10000 &40 &58530 &6503 \\
Science &10000 &66 &12576 &10000 &66 &12576 &1397 &10000 &66 &12516 &1388 \\
Sport &10000 &34 &47251 &10000 &34 &47251 &5250 &10000 &34 &46717 &5190 \\
Taxonomy &10000 &59 &18921 &10000 &59 &18921 &2102 &10000 &59 &19416 &2157 \\
\bottomrule
\end{tabular}
\end{adjustbox}
\end{table}


%%% TODO: Dataset statistics table
\begin{table}[!ht]
\centering
    \caption{Statistics of 3 transductive datasets}
    \begin{adjustbox}{max width=0.48\textwidth}
        \begin{tabular}{llcccc}
            \toprule
            \bf{Split} & \bf{Query Type} & \bf{FB15k} & \bf{FB15k-237} & \bf{NELL995} \\
            \midrule
            \multirow{2}{*}{Train}
            & 1p/2p/3p/2i/3i & 273,710 & 149,689 & 107,982 \\
            & 2in/3in/inp/pin/pni & 27,371 & 14,968 & 10,798 \\
            \midrule
            \multirow{2}{*}{Valid}
            & 1p & 59,078 & 20,094 & 16,910 \\
            & Others & 8,000 & 5,000 & 4,000 \\
            \midrule
            \multirow{2}{*}{Test}
            & 1p & 66,990 & 22,804 & 17,021 \\
            & Others & 8,000 & 5,000 & 4,000 \\
            \bottomrule
        \end{tabular}
    \end{adjustbox}
    \label{tab:app_transd_queries}
%
\centering
\caption{Statistics of 9 inductive $(e)$ datasets.}
\label{tab:app_inductive_e_queries}
%\scriptsize
\begin{adjustbox}{width=\textwidth}
\begin{tabular}{lrrrrrrrrrrrrrrrr}\toprule
Ratio & Graph & \multicolumn{1}{c}{\textbf{1p}} & \multicolumn{1}{c}{\textbf{2p}} & \multicolumn{1}{c}{\textbf{3p}} & \multicolumn{1}{c}{\textbf{2i}} & \multicolumn{1}{c}{\textbf{3i}} & \multicolumn{1}{c}{\textbf{pi}} & \multicolumn{1}{c}{\textbf{ip}} & \multicolumn{1}{c}{\textbf{2u}} & \multicolumn{1}{c}{\textbf{up}} & \multicolumn{1}{c}{\textbf{2in}} & \multicolumn{1}{c}{\textbf{3in}} & \multicolumn{1}{c}{\textbf{inp}} & \multicolumn{1}{c}{\textbf{pin}} & \multicolumn{1}{c}{\textbf{pni}} \\\midrule
\multirow{3}{*}{106\%} &training & 135,613 &50,000 &50,000 &50,000 &50,000 &50,000 &50,000 &50,000 &50,000 &50,000 &40,000 &50,000 &50,000 &50,000 \\
&validation & 6,582 &10,000 &10,000 &10,000 &10,000 &10,000 &10,000 &10,000 &10,000 &1,000 &1,000 &1,000 &1,000 &1,000 \\
&test & 5,446 &10,000 &10,000 &10,000 &10,000 &10,000 &10,000 &10,000 &10,000 &1,000 &1,000 &1,000 &1,000 &1,000 \\ \midrule
\multirow{3}{*}{113\%} &training & 115,523 &50,000 &50,000 &50,000 &50,000 &50,000 &50,000 &50,000 &50,000 &50,000 &40,000 &50,000 &50,000 &50,000 \\
&validation & 10,256 &10,000 &10,000 &10,000 &10,000 &10,000 &10,000 &10,000 &10,000 &1,000 &1,000 &1,000 &1,000 &1,000 \\
&test & 9,782 &10,000 &10,000 &10,000 &10,000 &10,000 &10,000 &10,000 &10,000 &1,000 &1,000 &1,000 &1,000 &1,000 \\ \midrule
\multirow{3}{*}{121\%} &training & 91,228 &50,000 &50,000 &50,000 &50,000 &50,000 &50,000 &50,000 &50,000 &50,000 &40,000 &50,000 &50,000 &50,000 \\
&validation & 12,696 &10,000 &10,000 &10,000 &10,000 &10,000 &10,000 &10,000 &10,000 &5,000 &5,000 &5,000 &5,000 &5,000 \\
&test & 14,458 &10,000 &10,000 &10,000 &10,000 &10,000 &10,000 &10,000 &10,000 &5,000 &5,000 &5,000 &5,000 &5,000 \\ \midrule
\multirow{3}{*}{133\%} &training & 75,326 &50,000 &50,000 &50,000 &50,000 &50,000 &50,000 &50,000 &50,000 &50,000 &40,000 &50,000 &50,000 &50,000 \\
&validation & 15,541 &50,000 &50,000 &50,000 &50,000 &50,000 &50,000 &20,000 &20,000 &5,000 &5,000 &5,000 &5,000 &5,000 \\
&test & 15,270 &50,000 &50,000 &50,000 &50,000 &50,000 &50,000 &20,000 &20,000 &5,000 &5,000 &5,000 &5,000 &5,000 \\ \midrule
\multirow{3}{*}{150\%} &training & 56,114 &50,000 &50,000 &50,000 &50,000 &50,000 &50,000 &50,000 &50,000 &50,000 &40,000 &50,000 &50,000 &50,000 \\
&validation & 16,229 &50,000 &50,000 &50,000 &50,000 &50,000 &50,000 &50,000 &50,000 &5,000 &5,000 &5,000 &5,000 &5,000 \\
&test & 17,683 &50,000 &50,000 &50,000 &50,000 &50,000 &50,000 &50,000 &50,000 &5,000 &5,000 &5,000 &5,000 &5,000 \\ \midrule
\multirow{3}{*}{175\%} &training & 38,851 &50,000 &50,000 &50,000 &50,000 &50,000 &50,000 &50,000 &50,000 &50,000 &40,000 &50,000 &50,000 &50,000 \\
&validation & 17,235 &50,000 &50,000 &50,000 &50,000 &50,000 &50,000 &50,000 &50,000 &10,000 &10,000 &10,000 &10,000 &10,000 \\
&test & 17,476 &50,000 &50,000 &50,000 &50,000 &50,000 &50,000 &50,000 &50,000 &10,000 &10,000 &10,000 &10,000 &10,000 \\ \midrule
\multirow{3}{*}{217\%} & training & 22,422 &30,000 &30,000 &50,000 &50,000 &50,000 &50,000 &50,000 &50,000 &30,000 &30,000 &50,000 &50,000 &50,000 \\
&validation & 18,168 &50,000 &50,000 &50,000 &50,000 &50,000 &50,000 &50,000 &50,000 &10,000 &10,000 &10,000 &10,000 &10,000 \\
&test & 16,902 &50,000 &50,000 &50,000 &50,000 &50,000 &50,000 &50,000 &50,000 &10,000 &10,000 &10,000 &10,000 &10,000 \\ \midrule
\multirow{3}{*}{300\%} &training & 11,699 &15,000 &15,000 &40,000 &40,000 &50,000 &50,000 &50,000 &50,000 &15,000 &15,000 &50,000 &40,000 &50,000 \\
&validation & 16,189 &50,000 &50,000 &50,000 &50,000 &50,000 &50,000 &50,000 &50,000 &10,000 &10,000 &10,000 &10,000 &10,000 \\
&test & 17,105 &50,000 &50,000 &50,000 &50,000 &50,000 &50,000 &50,000 &50,000 &10,000 &10,000 &10,000 &10,000 &10,000 \\ \midrule
\multirow{3}{*}{550\%} &training & 3,284 &15,000 &15,000 &40,000 &40,000 &50,000 &50,000 &50,000 &50,000 &10,000 &10,000 &30,000 &30,000 &30,000 \\
&validation & 13,616 &50,000 &50,000 &50,000 &50,000 &50,000 &50,000 &50,000 &50,000 &10,000 &10,000 &10,000 &10,000 &10,000 \\
&test & 13,670 &50,000 &50,000 &50,000 &50,000 &50,000 &50,000 &50,000 &50,000 &10,000 &10,000 &10,000 &10,000 &10,000 \\ 
\bottomrule
\end{tabular}
\end{adjustbox}
%
\centering
\caption{WikiTopics-CLQA statistics: the number of queries generated per query pattern for each topic knowledge graph of WikiTopics~\citep{isdea}. Numbers are the same for both the training and inference graph.  We follow the same 14 query patterns introduced by~\citet{betae}.}
%\scriptsize
\begin{adjustbox}{width=\textwidth}
\begin{tabular}{lrrrrrrrrrrrrrrr}\toprule
Topics & \bf{1p} & \bf{2p} & \bf{3p} & \bf{2i} & \bf{3i} & \bf{pi} & \bf{ip} & \bf{2in} & \bf{3in} & \bf{pin} & \bf{pni} & \bf{inp} & \bf{2u} & \bf{up} \\
\midrule
Art & 3113 & 10000 & 10000 & 10000 & 10000 & 10000 & 10000 & 1000 & 1000 & 1000 & 1000 & 1000 & 10000 & 10000 \\
Award & 2783 & 10000 & 10000 & 10000 & 10000 & 10000 & 10000 & 1000 & 1000 & 1000 & 1000 & 1000 & 10000 & 10000 \\
Education & 1575 & 10000 & 10000 & 10000 & 10000 & 10000 & 10000 & 1000 & 1000 & 1000 & 1000 & 1000 & 10000 & 10000 \\
Health & 1703 & 10000 & 10000 & 10000 & 10000 & 10000 & 10000 & 1000 & 1000 & 1000 & 1000 & 1000 & 10000 & 10000 \\
Infrastructure & 2405 & 10000 & 10000 & 10000 & 10000 & 10000 & 10000 & 1000 & 1000 & 1000 & 1000 & 1000 & 10000 & 10000 \\
Location & 8000 & 8917 & 4000 & 8000 & 8000 & 8000 & 8000 & 1000 & 1000 & 1000 & 1000 & 1000 & 8000 & 8000 \\
Organization & 3357 & 8000 & 4000 & 8000 & 8000 & 8000 & 8000 & 1000 & 1000 & 1000 & 1000 & 1000 & 8000 & 8000 \\
People & 6503 & 10000 & 10000 & 10000 & 10000 & 10000 & 10000 & 1000 & 1000 & 1000 & 1000 & 1000 & 10000 & 10000 \\
Science & 1388 & 10000 & 10000 & 10000 & 10000 & 10000 & 10000 & 1000 & 1000 & 1000 & 1000 & 1000 & 10000 & 10000 \\
Sport & 5190 & 8000 & 4000 & 8000 & 8000 & 8000 & 8000 & 1000 & 1000 & 1000 & 1000 & 1000 & 8000 & 8000 \\
Taxonomy & 2157 & 8000 & 8000 & 8000 & 8000 & 8000 & 8000 & 1000 & 1000 & 1000 & 1000 & 1000 & 8000 & 8000 \\
\bottomrule
\end{tabular}
\end{adjustbox}
\label{tab:wikitopics-clqa}
\end{table}

\subsection{WikiTopics-CLQA}
\label{app:subsec_wikitopcs_clqa}

The WikiTopics dataset introduced by~\citet{isdea} was used to evaluate link prediction model's zero-shot performance in the  inductive $(e,r)$ setting, i.e., when the test-time inference graph contains \textit{both} new entities and new relations unseen in training. It grouped relations into 11 different topics, or domains, such as art, education, health care, and sport. Two graphs, $\gtrain^{(T)}$ and $\ginf^{(T)}$ along with the missing triples $\edges^{(T)}_{\textit{valid}}$ and $\edges^{(T)}_{\textit{test}}$, were provided for each topic $T$, which had the same set of relations but different (potentially overlapping) set of entities. 
The goal was to train models on the training graphs $\gtrain^{(T)}$ of some topic $T$, and test on the inference graph $\ginf^{(T')}$ of an unseen topic $T'$. 
The model's validation performance was evaluated on the missing triples $\edges^{(T)}_{\textit{valid}}$ when observing training graph $\gtrain^{(T)}$ as inputs, and its test performance was evaluated on $\edges^{(T')}_{\textit{test}}$ when observing the test inference graph $\ginf^{(T')}$ as inputs. \Cref{tab:app_datasets_indr} shows the statistics of the 11 topic-specific knowledge graphs in WikiTopics.

We follow the procedures in BetaE~\citep{betae} to generate queries and answers of the 14 query patterns using the knowledge graphs in WikiTopics. We name the resulting dataset WikiTopics-CLQA. 
% We generate queries and answers for the training graph $\gtrain^{(T)}$ and inference graph $\ginf^{(T)}$ separately for each topic $T$
% , and we treat the label split, $\gG_{\textit{train-label}}^{(T)}$ and $\gG_{\textit{inf-label}}^{(T)}$, as the missing links, from which we extract the \textit{hard} answers. That is, entities that cannot be achieved by traversing the graph unless the missing link is predicted. 
For each topic $T$, we generate three sets of queries and answers (training, validation, test), using the training graph $\gtrain^{(T)}$ for training and validation, and inference graph $\ginf^{(T)}$ for test queries, respectively. 
Training queries on $\gtrain^{(T)}$ only have easy answers, 
validation (test) set easy answers are attained by traversing  $\gtrain^{(T)}$ ($\ginf^{(T)}$), whereas the full set of answers (easy and hard) are attained by traversing the graph $\gtrain^{(T)}$ merged with $\edges^{(T)}_{\textit{valid}}$ ($\ginf^{(T)}$ merged with $\edges^{(T)}_{\textit{test}}$). Hence, the hard answers cannot be found unless the model is capable of imputing missing links.
In our experiments, we only use the inference graph $\ginf^{(T)}$ and the test queries and answers for evaluating zero-shot inference performance.
%
\Cref{tab:wikitopics-clqa} shows the statistics of the WikiTopics-CLQA dataset. 

\section{Hyperparameters and Baselines}
\label{app:hyperparams}

Both \method and \methodlp are implemented with PyTorch~\cite{pytorch} (BSD-style license) and PyTorch-Geometric~\cite{pyg} (MIT license).
Both \method and \methodlp are initialized from the pre-trained \ultra checkpoint published by the authors and have the same GNN architectures with parameter count (177k).
\method is further trained for 10,000 steps on complex queries from the FB15k237 dataset. 
Hyperparameters of \method and its training details are listed in \autoref{tab:ultraquery_hparams}.

\methodlp employs thresholding of intermediate fuzzy set scores as one of the ways to alleviate the multi-source propagation issue (\Cref{subsec:ultra_proj}). 
Generally, the threshold is set to 0.8 with a few exceptions:
\begin{itemize}[leftmargin=*]
    \item 0.97 in NELL995
\end{itemize}

Below, we discuss the best available baselines for each dataset family.

\textbf{Transductive} (3 datasets): QTO~\citep{qto}. QTO requires  2000$d$ ComplEx~\citep{complex} entity and relation embeddings pre-computed for each graph, \eg, taking 30M parameters on the FB15k237 graph with 15k nodes. 
Further, QTO materializes the whole $(\gV \times \gV \times \gR)$ 3D tensor of scores of all possible triples for each graph. 
Pre-computing such tensors on three datasets takes considerable space and time, \eg, 8 hours for FB15k with heavy sparsification settings to fit onto a 24 GB GPU.

\textbf{Inductive $(e)$} (9 datasets): GNN-QE~\citep{gnn_qe}. The framework of GNN-QE is similar to \method but the backbone relation projection operator is implemented with NBFNet~\citep{nbfnet} which is only inductive to entities and still learns graph-specific relation embeddings. 
Parameter count, therefore, depends on the number of unique relation types, \eg, 2M for the FB 175\% split with 436 unique relations.

\textbf{Inductive $(e,r)$} (11 datasets): for newly sampled datasets, due to the absence of fully inductive trainable baselines, we compare against a rule-based heuristic baseline similar to the baseline in \citet{galkin2022}.
The baseline looks up candidate entities that have the same incoming relation type as the current relation projection (note that the identity of the starting head node in this case is ignored). 
The heuristic filters entities into two classes (satisfying the incoming relation type or not), hence, in order to create a ranking, we randomly shuffle entities in each class.
This baseline is non-parametric, does not require any training, and represents a sanity check of \clqa models. Still, as shown in \citet{galkin2022}, the baseline might outperform certain inductive reasoning approaches parameterized by neural networks. 


\begin{table}[t]
\centering
\caption{\method hyperparameters. $\text{GNN}_r$ denotes a GNN over the graph of relations $\grel$, $\text{GNN}_e$ is a GNN over the original entity graph $\gG$.}
\label{tab:ultraquery_hparams}
%\scriptsize
\begin{tabular}{lccc}\toprule
\multicolumn{2}{l}{Hyperparameter} &\method training \\\midrule
\multirow{4}{*}{$\text{GNN}_r$} &\# layers &6 \\
&hidden dim &64 \\
&message &DistMult \\
&aggeregation &sum \\ \midrule
\multirow{5}{*}{$\text{GNN}_e$} &\# layers &6 \\
&hidden dim &64 \\
&message &DistMult \\
&aggregation &sum \\
&$g(\cdot)$ &2-layer MLP \\ \midrule
\multirow{7}{*}{Learning} &optimizer &AdamW \\
&learning rate &0.0005 \\
&training steps &10,000 \\
&adv temperature & 0.2 \\
&traversal dropout & 0.25 \\
&batch size &64 \\
&training queries &FB15k237 \\
\bottomrule
\end{tabular}
\end{table}

\section{More Results}
\label{app:more_results}

\autoref{tab:app_full_results} corresponds to \autoref{fig:main_fig1} and \autoref{tab:maintab1} and provides MRR and Hits@10 results for \method, \methodlp, and Best Baseline for each dataset averaged across 9 EPFO query types and 5 negation query types. 
\autoref{fig:app_inde_full} is the full version of \autoref{fig:abl_multisource} and illustrates the performance of all 3 compared models on 9 inductive $(e)$ datasets on 14 query types together with their averaged values (EPFO avg and neg avg, respectively).

%%% ULTRA-Query More Results %%%
\begin{table}[!ht]
\centering
\caption{Full results (MRR, Hits@10) of \methodlp and \method in the zero-shot inference regime on transductive, entity-inductive $(e)$, and fully inductive $(e, r)$ datasets compared to the best-reported baselines averaged across 9 EPFO query types (EPFO avg) and 5 negation query types (Negation avg). \method was fine-tuned only on FB15k237 queries. The numbers correspond to~\Cref{tab:maintab1} and \Cref{fig:main_fig1}.}
\label{tab:app_full_results}
\begin{adjustbox}{width=\textwidth}
\begin{tabular}{lcccccccccccc}\toprule
&\multicolumn{4}{c}{\bf{\methodlp}} &\multicolumn{4}{c}{\bf{\method}} &\multicolumn{4}{c}{\bf{Best Baseline}}  \\  \cmidrule(l){2-5} \cmidrule(l){6-9} \cmidrule(l){10-13}
&\multicolumn{2}{c}{\bf{EPFO avg}} &\multicolumn{2}{c}{\bf{Negation avg}} &\multicolumn{2}{c}{\bf{EPFO avg}} &\multicolumn{2}{c}{\bf{Negation avg}} &\multicolumn{2}{c}{\bf{EPFO avg}} &\multicolumn{2}{c}{\bf{Negation avg}} \\ \cmidrule(l){2-3} \cmidrule(l){2-3} \cmidrule(l){4-5}  \cmidrule(l){6-7} \cmidrule(l){8-9} \cmidrule(l){10-11} \cmidrule(l){12-13} 
&MRR &Hits@10 &MRR &Hits@10  &MRR &Hits@10  &MRR &Hits@10  &MRR &Hits@10  &MRR &Hits@10 \\ \midrule
\multicolumn{13}{c}{transductive datasets} \\ \midrule
FB15k237 &0.216 &0.362 &0.082 &0.164 &0.242 &0.378 &0.08 &0.174 &0.335 &0.491 &0.155 &0.295 \\
FB15k &0.501 &0.672 &0.291 &0.465 &0.764 &0.834 &0.567 &0.725 &0.74 &0.837 &0.492 &0.664 \\
NELL995 &0.249 &0.395 &0.079 &0.16 &0.226 &0.341 &0.073 &0.159 &0.329 &0.483 &0.129 &0.268 \\ \midrule
\multicolumn{13}{c}{inductive $(e)$ datasets} \\ \midrule
FB 550\% &0.340 &0.518 &0.134 &0.251 &0.373 &0.535 &0.186 &0.332 &0.222 &0.331 &0.091 &0.158 \\
FB 300\% &0.320 &0.496 &0.117 &0.227 &0.359 &0.526 &0.168 &0.312 &0.291 &0.426 &0.125 &0.224 \\
FB 217\% &0.332 &0.509 &0.133 &0.252 &0.375 &0.537 &0.186 &0.337 &0.346 &0.492 &0.174 &0.301 \\
FB 175\% &0.297 &0.469 &0.110 &0.214 &0.338 &0.499 &0.159 &0.297 &0.351 &0.507 &0.188 &0.336 \\
FB 150\% &0.279 &0.445 &0.097 &0.190 &0.316 &0.473 &0.132 &0.255 &0.339 &0.493 &0.167 &0.303 \\
FB 133\% &0.266 &0.426 &0.087 &0.167 &0.298 &0.451 &0.122 &0.238 &0.353 &0.514 &0.197 &0.341 \\
FB 121\% &0.246 &0.400 &0.081 &0.164 &0.279 &0.430 &0.119 &0.232 &0.323 &0.462 &0.173 &0.291 \\
FB 113\% &0.217 &0.362 &0.067 &0.136 &0.240 &0.387 &0.097 &0.192 &0.352 &0.494 &0.214 &0.339 \\
FB 106\% &0.200 &0.340 &0.054 &0.114 &0.226 &0.370 &0.086 &0.162 &0.373 &0.504 &0.256 &0.377 \\ \midrule
\multicolumn{13}{c}{inductive $(e, r)$ datasets} \\ \midrule
Art &0.249 &0.389 &0.086 &0.157 &0.248 &0.349 &0.083 &0.137 &0.016 &0.031 &0.006 &0.014 \\
Award &0.224 &0.413 &0.046 &0.098 &0.227 &0.354 &0.152 &0.274 &0.004 &0.006 &0.002 &0.002 \\
Edu &0.142 &0.258 &0.066 &0.122 &0.179 &0.249 &0.119 &0.176 &0.008 &0.014 &0.003 &0.005 \\
Health &0.317 &0.466 &0.159 &0.231 &0.317 &0.394 &0.419 &0.525 &0.019 &0.040 &0.010 &0.020 \\
Infrastructure &0.392 &0.551 &0.087 &0.170 &0.337 &0.461 &0.235 &0.356 &0.010 &0.018 &0.003 &0.005 \\
Location &0.508 &0.678 &0.198 &0.371 &0.569 &0.679 &0.402 &0.585 &0.007 &0.017 &0.001 &0.002 \\
Organization &0.098 &0.190 &0.023 &0.048 &0.169 &0.270 &0.082 &0.171 &0.005 &0.008 &0.001 &0.002 \\
People &0.373 &0.530 &0.182 &0.281 &0.332 &0.443 &0.194 &0.285 &0.005 &0.010 &0.002 &0.003 \\
Science &0.206 &0.348 &0.048 &0.093 &0.158 &0.255 &0.071 &0.131 &0.041 &0.085 &0.010 &0.020 \\
Sport &0.215 &0.357 &0.095 &0.166 &0.252 &0.371 &0.178 &0.269 &0.015 &0.030 &0.005 &0.008 \\
Taxonomy &0.230 &0.315 &0.151 &0.255 &0.290 &0.360 &0.183 &0.261 &0.034 &0.066 &0.009 &0.017 \\
\bottomrule
\end{tabular}
\end{adjustbox}
\end{table}

\begin{figure*}[t]
    \centering
    \includegraphics[width=\linewidth]{figs/app_inde_pyg.pdf}
    \caption{Full results on 9 inductive $(e)$ datasets corresponding to \autoref{fig:abl_multisource}: albeit \methodlp outperforms the main \method on simple \emph{1p} queries, it suffers from the multi-source propagation issue on complex queries. \method trades a fraction of 1p query performance for significantly better average performance on 9 EPFO and 5 negation query types with particularly noticeable gains on intersection and \emph{2in}, \emph{3in} queries. }
    \label{fig:app_inde_full}
\end{figure*}


%%%% Multi-source propagation issue %%%
% \section{More Analysis on Multi-Source Propagation Issue}

%\input{sections/checklist}
%%%%%%%%%%%%%%%%%%%%%%%%%%%%%%%%%%%%%%%%%%%%%%%%%%%%%%%%%%%%%%%%%%%%%%%%%%%%%%%
%%%%%%%%%%%%%%%%%%%%%%%%%%%%%%%%%%%%%%%%%%%%%%%%%%%%%%%%%%%%%%%%%%%%%%%%%%%%%%%


\end{document}


% This document was modified from the file originally made available by
% Pat Langley and Andrea Danyluk for ICML-2K. This version was created
% by Iain Murray in 2018, and modified by Alexandre Bouchard in
% 2019 and 2021 and by Csaba Szepesvari, Gang Niu and Sivan Sabato in 2022.
% Modified again in 2023 and 2024 by Sivan Sabato and Jonathan Scarlett.
% Previous contributors include Dan Roy, Lise Getoor and Tobias
% Scheffer, which was slightly modified from the 2010 version by
% Thorsten Joachims & Johannes Fuernkranz, slightly modified from the
% 2009 version by Kiri Wagstaff and Sam Roweis's 2008 version, which is
% slightly modified from Prasad Tadepalli's 2007 version which is a
% lightly changed version of the previous year's version by Andrew
% Moore, which was in turn edited from those of Kristian Kersting and
% Codrina Lauth. Alex Smola contributed to the algorithmic style files.
