\section{Background: An Overview of PRISM}
\label{sec:prism}

%PAGE BUDGET: 1.0 page including Extended PRISM.

PRISM programs have Prolog-like syntax (see Example~\ref{ex:fmm}).
In a PRISM program the \texttt{msw} relation (``multi-valued switch'')
has a special meaning: \texttt{msw(X,I,V)} says that \texttt{V} is a
random variable.  More precisely, \texttt{V} is the outcome of the
\texttt{I}-th instance from a family \texttt{X} of random
processes\footnote{Following PRISM, we often omit the instance number
  in an \texttt{msw} when a program uses only one instance from a
  family of random processes.}.  The set of variables $\{\mathtt{V}_i
\mid \mathtt{msw(}p, i, \mathtt{V}_i\mathtt{)}\}$ are i.i.d., and
their distribution is given by the random process $p$.  The
\texttt{msw} relation provides the mechanism for using random
variables, thereby allowing us to weave together statistical and
logical aspects of a model into a single program.  The distribution
parameters of the random variables are specified separately.


PRISM programs have declarative semantics, called 
\emph{distribution semantics}~\cite{sato-kameya-prism,sato}.
Operationally,  query evaluation in PRISM closely follows that for
traditional logic programming, with one
modification.  When the goal selected at a step is of the form
\texttt{msw(X,I,Y)}, then \texttt{Y} is bound to a possible outcome of a
random process \texttt{X}.  The derivation step is
associated with the probability of this outcome.  If all random
processes encountered in a derivation are independent, then the
probability of the derivation is the product of probabilities of each
step in the derivation.  If a set of derivations are pairwise mutually
exclusive, the probability of the set is the sum of probabilities of
each derivation in the set\footnote{The evaluation procedure is
  defined only when the independence and exclusiveness assumptions
  hold.}.  
Finally, the probability of an answer to
a query is computed as the probability of the set of derivations
corresponding to that answer.  

As an illustration, consider the query \texttt{fmix(X)} evaluated over program in
Example~\ref{ex:fmm}.   One step of resolution derives goal of
the form \texttt{msw(m, M), msw(w(M),X)}. Now depending on the value of \texttt{m},
there are two possible next steps:
\texttt{msw(w(a),X)} and \texttt{msw(w(b),X)}.
\emph{Thus in PRISM, derivations are constructed by enumerating the
  possible outcomes of each random variable.}


\begin{Ex}[Finite Mixture Model]
In the following PRISM program, which encodes a finite mixture model~\cite{fmm},
{\rm \texttt{msw(m, M)}} chooses one distribution from a finite set of
continuous 
distributions, {\rm \texttt{msw(w(M), X)}} samples {\rm \texttt{X}} from the chosen distribution.

\begin{small}
  \begin{minipage}{3.0in}
\begin{verbatim}
fmix(X) :- msw(m, M),
           msw(w(M), X).

% Ranges of RVs
values(m, [a,b]).
values(w(M), real).
% PDFs and PMFs
:- set_sw(m, [0.3, 0.7]),
   set_sw(w(a), norm(2.0, 1.0)),
   set_sw(w(b), norm(3.0, 1.0)).
\end{verbatim}
\end{minipage}
\end{small}
\qed
  \label{ex:fmm}
\end{Ex}


\section{Extended PRISM}
\label{sec:language}

Support for continuous variables is added by modifying PRISM's
language in two ways.   We use the \texttt{msw}
relation to sample from discrete as well as continuous distributions.
In PRISM, a special relation
called \texttt{values} is used to specify the ranges of values of
random variables; the probability mass functions are specified using
\texttt{set\_sw} directives.  We extend the
\texttt{set\_sw} directives to specify probability density functions
as well.  For instance, \texttt{set\_sw(r, norm(Mu,Var))} specifies
that outcomes of  random processes \texttt{r} have Gaussian
distribution with mean 
\texttt{Mu} and variance \texttt{Var}.
\comment{ \footnote{The technical
  development in this paper considers only univariate Gaussian
  variables; see Discussions section on a discussion on how
  multivariate Gaussian as well as other continuous distributions are
  handled.}}  Parameterized families of random
processes may be specified, as long as the parameters are
discrete-valued.  For 
instance, \texttt{set\_sw(w(M), norm(Mu,Var))} specifies a family of
random processes, with one for each value of \texttt{M}. As
in PRISM, \texttt{set\_sw} directives may be specified
programmatically; for instance, the distribution parameters of \texttt{w(M)},
 may be computed as functions of
\texttt{M}.

Additionally, we extend PRISM programs with linear equality
constraints over reals.  Without loss of generality, we
assume that constraints are written as linear equalities of the form
$Y = a_1 * X_1 + \ldots + a_n * X_n + b$ where $a_i$ and $b$ are all
floating-point constants.
\comment{ , or inequalities of the form $Y < a$ or $Y>a$
for some floating point constant $a$. 
Note that inequalities comparing
two variables can be expressed as an inequality comparing a linear
function of the two variables and a constant.}
The use of constraints enables us to encode Hybrid Bayesian Networks and Kalman Filters as
extended PRISM programs.  In the following, we use \emph{Constr} to
denote a set (conjunction) of linear equality 
constraints.  We also denote by $\overline{X}$ a vector of variables
and/or values, explicitly specifying the size only when it is not
clear from the context.  This permits us to write linear equality
constraints compactly (e.g., $Y = \overline{a}\cdot\overline{X} +b$).



%%% Local Variables:
%%% mode: latex
%%% TeX-master: "main"
%%% End:
