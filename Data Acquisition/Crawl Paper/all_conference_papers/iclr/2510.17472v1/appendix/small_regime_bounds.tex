Hoeffding, Bernstein, and Chernoff–Markov bounds become less effective in the small-sample regime, i.e., when the number of voters satisfies $n\lesssim 50$. In this setting, the exact error probability can be obtained by leveraging the properties of the  multinomial distribution.
We provide below an efficient dynamic programming (DP) approach to compute this probability.


\subsubsection{Dynamic programming for exact multinomial probabilities}\label{app:subsubsec_dynamic_programming}
For each category $j$, define 
$$
P_j(x) =\frac{p_j^{x}}{x!} , \quad x = 0, \dots, n,
$$
and store the values $P_j= (P_j(0),\dots, P_j(n))$ in an array of length $n+1$.
The entries can be generated iteratively using the recurrence 
$$P_j(x+1) = \frac{p_j}{(x+1)}P_j(x).$$

After processing a subset of the rival categories, we define a state of the dynamic program as
$$
  \text{state }(t,m,s)\quad
  \text{with}\quad
  \begin{cases}
     t &=\text{votes for the true category }c^\star,\\
     m &=\text{current {\it maximum} vote count among the rivals processed so far,}\\
    s &=\text{{\it total} vote count allocated to processed categories.}
  \end{cases}
$$
Formally, the DP table is
$$
  \text{DP}_{i}(t,m,s)
  \;=\;\frac{1}{s!}\,
  \mathbb{P}\Bigl[N_{c^\star}^{(s)}=t,\;
           \max_{j\in\{1,\dots,i\}\setminus\{c^\star\}}N_j^{(s)} = m, \; \sum_{j\in\{c^\star, 1, \dots, i\}}N_i^{(s)} =s\Bigr],
$$
where $i$ denotes the number of categories processed.

\paragraph{Initial table.}
Before incorporating any rivals, we only consider the true category $c^\star$
$$
  \text{DP}_{1}(t,0,t)=P_{c^\star}(t),\qquad
  t=0,\dots,n.
$$
since the maximum vote count among zero rivals is naturally $0$.

\paragraph{Transition when adding a new rival $j$.}
Suppose we have already computed $\text{DP}_{i}(\cdot,\cdot, \cdot)$. 
We now incorporate category $j$.
For each triple $(t, m, s)$, we consider vote counts  $x=0,\dots,n-s$ drawn from $P_j(x)$ and define
$$
  \text{newMax}=\max\{m,x\}.
$$
The DP table is updated according to
$$
  \text{DP}_{i+1}(t,\text{newMax}, s+x)
  \;+\!=\;
  \text{DP}_{i}(t,m,s)\;P_j(x).
$$

This implementation stores states $(t,m,s)$ with $t+m\leq s\leq n$, and transitions $x=0, \dots, n-s$. 
Because of these constraints, the total number of reachable states is $\mathcal{O}(n^{3})$, rather than the naive $\mathcal{O}(n^{4})$. Iterating over all $k$ categories therefore yields a worst-case time complexity of
$$\mathcal{O}(kn^{3}).$$
The memory complexity is $\mathcal{O}(n^{3})$.

After processing all $k-1$ rivals, the DP table $\text{DP}_{k}(\cdot,\cdot,\cdot)$ is complete.
The error probability is then obtained by summing over all states where the true category does not have a strict majority and the total number of votes is equal to $n$, 
$$
  \mathbb{P}(\hat{c}_n \neq c^\star)
  \;=\;
  \sum_{t=0}^{n}\,
  \sum_{m=t}^{n}
  n!\;\text{DP}_{k}(t,m,n).
$$
For large values of $n$, factorial terms may cause numerical underflow or overflow. To prevent this, we compute the entries of the DP table in log space.