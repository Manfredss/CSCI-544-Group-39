\begin{proof}[Proof]
Let $\mathbf{p} = (p_1, \dots, p_k)$ denote the true probability distribution, and let $\mathbf{\hat{p}_n} = (\hat{{p}}_{n,1}, \dots, \hat{{p}}_{n,k})$ be the empirical measure, where $\hat{p}_{n,j}$ are the empirical frequencies for each category
$$
\hat p_{n,j}=\frac1n\sum_{i=1}^n\mathbf 1\{X_i=j\}.
$$
Recall that $p_{c^\star} = \max_j p_j$.
Define the set $\mathcal{B}\subseteq\Delta_k$ by
\begin{equation}\label{eq:bad_set}
\mathcal B
=\left\{\mathbf q\in\Delta_k:\;q_{c^\star}\le\max_{j \neq c^\star}q_j\right\}= \left\{ \mathbf q \in \Delta_k : q_{c^\star} \leq q_j, \mbox{ for some } j \neq c^\star\right\}.    
\end{equation}

    \paragraph{Step 1. Sanov upper bound.} 
Sanov’s theorem (large--deviation principle for types) states that for
any Borel set $\mathcal A\subseteq\Delta_k$,
$$
-\inf_{\mathbf{q}\in\mathring{\mathcal A}}D_{\mathrm{KL}}(\mathbf{q}\|\mathbf p)
\le
\liminf_{n\to\infty}\frac1n\log
\mathbb P\left(\hat{\mathbf p}_n\in\mathcal A\right)
\le
\limsup_{n\to\infty}\frac1n\log
\mathbb P\left(\hat{\mathbf p}_n\in\mathcal A\right)
\le
-\inf_{\mathbf{q}\in\overline{\mathcal A}}D_{\mathrm{KL}}(\mathbf q\|\mathbf p),
$$
where $\mathring{\mathcal A}$ and $\overline{\mathcal A}$ denote the interior and closure of $\mathcal{A}$, respectively. 

For our purposes, let $\mathcal{A} = \mathcal{B}$ as defined in Eq. (\ref{eq:bad_set}). Then
$$
\mathring{\mathcal B} = \{\mathbf q \in \Delta_k : q_{c^\star} < q_j \text{ for some } j \neq c^\star \}, 
\quad \overline{\mathcal B} = \mathcal B,
$$
since $\mathcal{B}$ is closed.

    \paragraph{Step 2. Error event as a type set.}
    The majority rule is \emph{incorrect} (i.e. $\hat{c}_n = \arg\max_j \,\hat{{p}}_{n,j}\neq c^\star$) 
    exactly when $\hat{\mathbf p}_n\in\mathcal B$. 
Hence, applying Sanov’s bounds yields
$$
\mathbb P\left[\hat{c}_n\neq c^\star\right] = 
\exp\left(-n\,\inf_{\mathbf q\in\mathcal B}D_{\mathrm{KL}}(\mathbf q\|\mathbf p)
+o(n)\right).
$$
    \paragraph{Step 3. Positivity of the exponent.}
    If $p_{c^\star}>p_j$ for every $j \neq c^\star$, then the true distribution satisfies $\mathbf p\notin\mathcal B$.
    The infimum of the KL divergence over $\mathcal B$ is therefore attained on the boundary, i.e., at some $\mathbf q^\star \in \mathcal B$ with $q^\star_{c^\star} = q^\star_{j}$ for some $j \neq c^\star$. 
Thus, the large-deviation exponent is
$$
I^\star (\mathbf{p})
=\min_{j \neq c^\star}\;
\inf_{\mathbf{q}:\,q_{c^\star}=q_j}D_{\mathrm{KL}}(\mathbf q\|\mathbf p)
>0,
$$
and the error probability decays exponentially
$$
\mathbb P[\hat{c}_n \neq c^\star] = \exp(-n I^\star(\mathbf p) + o(n)).
$$
\end{proof}
\subsubsection{Sanov exponent}
The Sanov exponent $I^\star(\mathbf{p})$ admits a closed-form expression. We provide a detailed derivation below.

Recall that our objective is to compute
\begin{equation}\label{eq:bad_set_sanov_exponent}
    I^\star(\mathbf{p}) = \inf_{\mathbf{q}\in\mathcal{B}} D_\text{KL}(\mathbf{q}\Vert \mathbf{p}),\quad\,\, \mathcal{B} = \{\mathbf{q}\in \Delta_k:\,q_{c^\star}\leq \max_{j \neq c^\star} q_j\}.
\end{equation}
Let the runner-up (second-largest competitor) be
$${j^\star} = \arg\max_{j \neq c^\star}\,\, p_j.$$ 
Then the optimisation problem can be equivalently written as
$$
I^\star(\mathbf{p}) = \min_{\mathbf q \in \Delta_k}\left[\sum_{i=1}^k q_i \log\left (\frac{q_i}{p_i}\right) \,: q_{j^\star} \geq q_{c^\star}\right].
$$
Introducing Lagrange multipliers, we define the Lagrangian
$$
\mathcal{L}(q, \lambda, \mu) = \sum_{i=1}^kq_i\log \frac{q_i}{p_i} + \lambda\left(\sum_{i=1}^k q_i - 1\right) + \mu(q_{c^\star} - q_{j^\star}),
$$
where $\lambda\in\mathbb{R}$ and $\mu\geq 0$, to enforce $q_{c^\star}\leq q_{j^\star}$. 
The first-order conditions yield
$$
\partial_{q_i}\mathcal{L} = \log q_i + 1 - \log p_i + \lambda = 0, \quad \,\, \text{for \,\,\,}i \neq c^\star, j^\star,
$$
which implies
$$
q_i = p_i e^{-(1 + \lambda)}, \quad \,\, \text{for \,\,\,}i \neq c^\star, j^\star.
$$
Similarly, for $q_{c^\star}$ and $q_{j^\star}$ we obtain
$$
\begin{aligned}
q_{c^\star} = p_{c^\star} e^{-(1+\lambda+\mu)},\qquad
q_{j^\star}=p_{j^\star}e^{-(1+\lambda-\mu)}.
\end{aligned}
$$
Defining $Z = e^{-(1+\lambda)}$ and $s = e^{\mu}$, we can rewrite the solution as
$$
\begin{aligned}
q_i &= p_i Z, \quad\,\, i \neq c^\star, j^\star\\
q_{c^\star} &= p_{c^\star} Z/s \\
q_{j^\star} &= p_{j^\star}Zs.
\end{aligned}
$$
Solving for $s$, we have
$$
s = \sqrt{\frac{q_{j^\star}}{p_{j^\star}} \frac{p_{c^\star}}{q_{c^\star}}}.
$$
Enforcing the constraint $q_{j^\star} \geq q_{c^\star}$ gives
$$
s \geq \sqrt{\frac{p_{c^\star}}{p_{j^\star}}}.
$$
On the other hand, enforcing the simplex constraint $\sum_i q_i = 1$ gives
$$
Z\left[(1 - p_{c^\star} - p_{j^\star}) + \frac{p_{c^\star}}{s} + p_{j^\star}s\right] = 1.
$$
    Note that $Z>0$. Substituting this, the KL divergence can be expressed as a function of $s$ (since $Z$ itself depends on $s$)
    \begin{align*}
    D_{\text{KL}}(\mathbf{q}(s)\Vert \mathbf{p} ) &= Z\log Z\left[(1-p_{c^\star}-p_{j^\star})+\frac{p_{c^\star}}{s}+p_{j^\star}s\right]
    +Z\log s\left(p_{j^\star}s-\frac{p_{c^\star}}{s}\right) 
    \\&= \log Z + Z \log s\left(p_{j^\star} s - \frac{p_{c^\star}}{s}\right).
    \end{align*}
    Minimising over $\mathbf q$ is equivalent to optimising over $s \geq \sqrt{p_{c^\star}/p_{j^\star}}$.   
    Focusing on the first term, we observe that
    $$
    \frac{d}{ds}\log Z = -\frac{-p_{c^\star}/s^2 + p_{j^\star}}{\left((1 - p_{c^\star} - p_{j^\star}) + p_{c^\star}/s + p_{j^\star}s\right)} \leq 0, \quad\mbox{ for } s \geq \sqrt{p_{c^\star}/p_{j^\star}}.
    $$
    Therefore, $\log Z$ is strictly decreasing for $s\in\left(\sqrt{p_{c^\star}/p_{j^\star}}, \infty\right)$. 

Furthermore, the derivative of the KL divergence with respect to $s$ is
\begin{align*}
\frac{d}{ds}D_{\text{KL}}(\mathbf{q}(s)\Vert \mathbf{p} ) &= Z \log s\left(p_{j^\star}  + \frac{p_{c^\star}}{s^2} -Zs\left(p_{j^\star}-\frac{p_{c^\star}}{s^2}\right)^2\right), \quad\,\, s \geq \sqrt{p_{c^\star}/p_{j^\star}}> 0.
\end{align*}
Note that 
\begin{align*}
 s\left[\left(p_{j^\star}+\frac{p_{c^\star}}{s^{2}}\right)
- Z\left(p_{j^\star}-\frac{p_{c^\star}}{s^2}\right)^2
\right] &\geq \left[\left(p_{j^\star}s+\frac{p_{c^\star}}{s}\right)
-
\frac{(p_{j^\star} s-p_{c^\star}/s)^{2}}{\,p_{j^\star} s + {p_{c^\star}}/{s}}
\right] \\&\geq \left[\left(p_{j^\star}s+\frac{p_{c^\star}}{s}\right)
\;-\;
\frac{(p_{j^\star} s+p_{c^\star}/s)^{2}}{\,p_{j^\star} s + p_{c^\star}/s}
\right]  \geq 0.
\end{align*}
In particular, for $s>\sqrt{p_{c^\star}/p_{j^\star}}$
$$
 s\left[\left(p_{j^\star}+\frac{p_{c^\star}}{s^{2}}\right)
- Z\left(p_{j^\star}-\frac{p_{c^\star}}{s^2}\right)^2
\right] >0.
$$
Using this together with the fact that $Z>0$ and $\log s>0$ (since $s> 1$), it follows that
$$
\frac{d}{ds}D_{\text{KL}}(\mathbf{q}(s)\Vert \mathbf{p} )>0, \quad\mbox{ for } s > \sqrt{p_{c^\star}/p_{j^\star}}.
$$
Hence, $D_{\text{KL}}(\mathbf{q}(s)\Vert \mathbf{p} )$ is strictly increasing for $s\in(\sqrt{p_{c^\star}/p_{j^\star}}, \infty)$. The minimum is therefore attained at $s = \sqrt{p_{c^\star}/p_{j^\star}}$, which leads to
\begin{align*}
I^\star(\mathbf{p}) &=\min_{\substack{\mathbf{q}\in\Delta_k\\q_{j^\star}\ge q_{c^\star}}} 
D_{\text{KL}}(\mathbf{q}\Vert \mathbf{p})
=-\log \left(1-p_{c^\star}-p_{j^\star}+2\sqrt{p_{c^\star} p_{j^\star}}\right) \\
&= -\log\left(1-\left(\sqrt{p_{c^\star}}-\sqrt{p_{j^\star}}\right)^2\right).
\end{align*}

\begin{remark}
\leavevmode
\begin{enumerate}
\item If there are multiple runners-up, there may be several minimisers $\mathbf q^\star$ in the set $\mathcal{B}$ defined in Eq. (\ref{eq:bad_set_sanov_exponent}), but they all yield the same KL value. Therefore $I^\star(\mathbf{p})$ remains unchanged regardless of the number of runners-up.
\item If $p_{c^\star} = p_{j^\star}$, then $I^\star(\mathbf{p}) = 0$.
\end{enumerate}
\end{remark}
\vspace{8pt}
\begin{remark}
By expanding the Sanov exponent in terms of the probability gap, we recover the error rate obtained via direct application of the central limit theorem. Specifically, for $\delta = p_{c^\star} - p_{j^\star} \ll p_{j^\star}$,
$$
I^\star(\mathbf{p}) = \frac{\delta^2}{4 p_{j^\star}}\left[1 + O(\delta/p_{j^\star})\right].
$$
On the other hand, since $\sigma_{j^\star}^2 = 2p_{j^\star} + \delta - \delta^2$, we have $
2 \sigma_{j^\star}^2 = 4 p_{j^\star} + O(\delta).
$
Therefore,
$$
\frac{\delta^2}{2\sigma_{j^\star}^2} = \frac{\delta^2}{4p_{j^*}}[1 + O(\delta/p_{j^*})],
$$
which yields
$$
I^\star(\mathbf{p}) = \frac{\delta^2}{2\sigma_{j^\star}^2} + O(\delta^3).
$$
\end{remark}

\subsubsection{Bahadur-Rao correction}
The random variable $Y_i^j = \mathbf 1\{X_i=c^\star\}-\mathbf 1\{X_i=j\} \in\{-1,0,1\}$ is integer-valued with span $d=1$ and has logarithmic moment generating function
$$
\Lambda_Y(\lambda) = \log M(\lambda) = \log \mathbb{E}\left[e^{\lambda Y_i^j}\right] = \log\left(p_{c^\star} e^{\lambda} + p_j e^{-\lambda} + 1- (p_{c^\star}+p_j)\right).
$$
Consider $S_j^{(n)}=-\sum_i^n Y_i^j = -Z_j^{(n)}$. We have that 
$$
\Lambda_{-Y}(\lambda) = \Lambda_Y(-\lambda). 
$$
Let $\lambda^\star =\tfrac{1}{2}\log(p_j/p_{c^\star}) < 0$ denote the minimiser of the moment generating function $M(\lambda)$ and define $\eta = -\lambda^\star>0$. 
Then, 
$$
\Lambda_{-Y}'(\eta) = -\Lambda'_Y(-\eta) = -\Lambda'_Y(\lambda^\star) =0.
$$
We are interested in the event $S_j^{(n)}\geq q$, where $q=\Lambda_{-Y}'(\eta)=0$. 
By \citet[Theorem 3.7.4. (b)]{dembo2010ldp}, as $n\to \infty$, we obtain the exact asymptotics 
$$
\mathbb{P}\left[S_j^{(n)}\geq nq\right] = \frac{1+o(1)}{(1-\exp(\lambda^\star))\sqrt{2\pi n\Lambda''_{-Y}(-\lambda^\star)}}\exp\left(-n\ \Lambda^\star_{-Y}(q)\right),
$$
where $\Lambda^\star_{-Y}(q)$ is the Legendre transform given by
\begin{align*}
 \Lambda^\star_{-Y}(q) &= \eta q - \Lambda_{-Y}(\eta) = -\Lambda_{-Y}(-\lambda^\star) = -\Lambda_{Y}(\lambda^\star)\\
 &=-\log\left(1-\left(\sqrt{p_{c^\star}}-\sqrt{p_j}\right)^2\right)
\end{align*}
and 
$$
\Lambda''_{-Y}(-\lambda^\star) = \Lambda_Y(\lambda^\star) = \frac{M''(\lambda^\star)}{M(\lambda^\star)} = \frac{2\sqrt{p_{c^\star} p_j}}{1-(\sqrt{p_{c^\star}}- \sqrt{p_j})^2}:=\tilde \sigma_j^2.
$$
Finally, we have that as $n\to\infty$,
\begin{align*}
    \mathbb{P}\bigl[\hat{c}_n&\neq c^\star\bigr]
    \le \sum_{j\neq c^\star}\mathbb{P}\bigl[Z_j^{(n)}\le 0\bigr]
    =\sum_{j\neq c^\star}\mathbb{P}\bigl[S_j^{(n)}\ge 0\bigr]\\
    &=(1+o(1))\sum_{j\neq c^\star}\frac{1}{\sqrt{2\pi n}(1-\exp(1/2\log(p_j/p_{c^\star})))\tilde\sigma_j}\exp\left(n\log(1-(\sqrt{p_{c^\star}}-\sqrt{p_j})^2)\right)\\
    &\sim \frac{1}{\sqrt{2\pi n}(1-\exp(1/2\log(p_{j^\star}/p_{c^\star})))\tilde\sigma_{j^\star}}\exp\left(n\log(1-(\sqrt{p_{c^\star}}-\sqrt{p_{j^\star}})^2)\right)\\
    &\sim \frac{1}{\sqrt{2\pi n}(1-\exp(1/2\log(p_{j^\star}/p_{c^\star})))\tilde\sigma_{j^\star}}\exp\left(-n I^\star(\mathbf{p})\right),
\end{align*}
where
${j^\star} = \arg\max_{j \neq c^\star}\,\, p_j$ is the runner-up.



