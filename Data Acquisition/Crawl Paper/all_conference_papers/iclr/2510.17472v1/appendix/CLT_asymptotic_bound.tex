\begin{proof}
For each rival $j$, consider the random variable $Z_j^{(n)}$ defined in (\ref{eq:random_variable_sum_auxiliary}), which is a sum of independent and identically distributed random variables with mean 
$$\mu_j = p_{c^\star}-p_j$$
and variance
$$
\sigma_j^2 = p_{c^\star}+p_j - (p_{c^\star}-p_j)^2.
$$
By the central limit theorem,
$$
\frac{Z_j^{(n)}-n(p_{c^\star}-p_j)}{\sqrt{n}}\xrightarrow{d} \mathcal{N}(0,\sigma_j^2).
$$
Therefore, as $n\to \infty$, we have
\small
$$
\mathbb P\left[Z_j^{(n)}\le 0\right]
= \mathbb P\left[\frac{Z_j^{(n)}-n(p_{c^\star}-p_j)}{\sigma_j\sqrt n}
                 \le -\,\frac{(p_{c^\star}-p_j)\sqrt n}{\sigma_j}\right]
= \Phi\Bigl(-\,\frac{(p_{c^\star}-p_j)\sqrt n}{\sigma_j}\Bigr)
  \,[1+O(n^{-1/2})],
$$
\normalsize
where $\Phi$ denotes the CDF of a standard Gaussian random variable.

Majority voting fails if $Z_j^{(n)}\le 0$ for some $j \neq c^\star$.
Applying the union bound, we obtain
$$
\mathbb P[\hat{c}_n\neq c^\star]
   \;\le\;\sum_{j \neq c^\star}\mathbb P\left[Z_j^{(n)}\le 0\right]
   \;=\;\sum_{j \neq c^\star}
        \Phi\Bigl(-\,\frac{(p_{c^\star} - p_j)\sqrt n}{\sigma_j}\Bigr)
        [1+O(n^{-1/2})].
$$
To bound the Gaussian tail, we use Craig's formula
$$
\Phi(-x) = \frac{1}{\pi}\int_0^{\pi/2}\exp\left(-\frac{x^2}{2\sin^2\theta}\right)d\theta\le \frac{1}{2}e^{-\frac{x^{2}}{2}}, \quad\,\,\text{for $x>0$}.
$$
Substituting this bound gives
$$
\mathbb P[\hat{c}_n\neq c^\star]
   \;\le\;\frac{1}{2}\sum_{j \neq c^\star} \exp\left(-\frac{n}{2}\left(\frac{p_{c^\star}-p_j}{\sigma_j}\right)^2\right)\leq \frac{1}{2}(k-1) \exp\left(-\frac{n}{2}\min_{j \neq c^\star}\left(\frac{p_{c^\star}-p_j}{\sigma_j}\right)^2\right).
$$
For fixed $p_{c^\star}$, the ratio $\frac{p_{c^\star}-p_j}{\sigma_j}$ is monotonically decreasing in $p_j$. 
Therefore, the smallest value, and hence the slowest exponential decay, is attained at the rival with the largest probability among the competitors. Denoting this \emph{second-largest} vote  probability by
$$p_{j^\star} = \max_{j \neq c^\star} p_j,$$
the convergence rate in the exponential bound above is determined by 
$$
\kappa = \frac{\delta}{2 p_{c^\star} - \delta-\delta^2},\quad\,\,\ \delta = p_{c^\star}-p_{j^\star}.
$$
Thus, the competitor that most threatens the accuracy of majority voting is precisely the category with the second–highest support.

The previous bound derived from the central limit theorem can be sharpened by incorporating two classical corrections.
The first correction is the continuity term, that is, the correction term due to discreteness. Since the random variable $Z_j^{(n)}$ takes values in the discrete set $\{-n, \dots, n\}$, the event $Z_j^{(n)}\leq 0$ is equivalent to $Z_j^{(n)}\leq 1/2$. Hence,
$$
\mathbb{P}\left[Z_j^{(n)}\leq 0\right] = \mathbb{P}\left[Z_j^{(n)}\le 1/2\right].
$$
Applying the central limit theorem approximation then yields, as $n\to\infty$,
\begin{align*}
\mathbb P\left[Z_j^{(n)}\le 1/2\right]
&= \mathbb P\left[\frac{Z_j^{(n)}-n(p_{c^\star}-p_j)}{\sigma_j\sqrt n}
                 \le\frac{1}{2\sigma_j\sqrt{n}} -\frac{\sqrt n(p_{c^\star}-p_j)}{\sigma_j}\right]\\
&\approx \Phi\left(\frac{\sqrt n(p_{c^\star}-p_j)-1/(2\sqrt{n})}{\sigma_j}\right) = \frac{1}{2}\text{erfc} \left(\frac{\sqrt n(p_{c^\star}-p_j)-1/(2\sqrt{n})}{\sqrt{2}\,\sigma_j}\right).
\end{align*}
A further refinement comes from the Berry–Esseen theorem, which quantifies the uniform error of the central limit theorem approximation. In particular, for all $n$
$$
\left\vert \mathbb{P}\left[\frac{Z_j^{(n)}-n(p_{c^\star}-p_j)}{\sigma_j\sqrt n}\le x\right] -\Phi(x)\right\vert \leq \frac{C \rho_j}{\sigma_j^3\sqrt{n}},
$$
where $\rho_j$ denotes the third central moment,
$$
\rho_j = \mathbb{E}\left[\left(Y_i^j-(p_{c^\star}-p_j)\right)^3\right] = (p_{c^\star} - p_{j})(1-3(p_{c^\star} + p_{j}) + 2(p_{c^\star} - p_{j})^2)
$$
and $C\le 0.56$ is a universal constant. Incorporating both corrections, we obtain the refined bound
$$
\mathbb P[\hat{c}_n\neq c^\star]\leq \sum_{j \neq c^\star}\frac{1}{2}\text{erfc}\left(\frac{{\sqrt{n}(p_{c^\star}-p_j)} - {1}/{(2\sqrt{n}})}{\sqrt{2}\,\sigma_j}\right) + \frac{0.56\rho_j}{\sigma_j^3\sqrt{n}}.
$$
\end{proof}